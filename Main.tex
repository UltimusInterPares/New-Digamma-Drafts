\documentclass[draft]{turabian-researchpaper}
%%% Turabian Formatting

% Access the turabian-researchpaper Document Class
\usepackage{turabian-formatting}

% Packages recommended by turabian-formatting
\usepackage{tocloft}
\usepackage{ellipsis}
\usepackage{threeparttable}
%%% Text Formatting

% Character Input
\usepackage[utf8]{inputenc}
		\DeclareUnicodeCharacter{1E5A}{\d{R}}
		%\DeclareUnicodeCharacter{01FD}{\'{\ae}}
	    %\DeclareUnicodeCharacter{1E31}{ḱ}
	    %\DeclareUnicodeCharacter{02B7}{ʷ}
	    \DeclareUnicodeCharacter{025B}{\ipa{E}}
	    \DeclareUnicodeCharacter{0254}{\ipa{O}}
	    \DeclareUnicodeCharacter{21B5}{\rotatebox[origin=c]{180}{$\Rsh$}}
    
% Gotta have that font encoding
\usepackage[T1]{fontenc}

% Nicer Setting
\usepackage{microtype}

% Coolors for links
\usepackage[table]{xcolor}

% Book Sections
\newcommand{\bookref}[3]{#1~{\MakeUppercase{\romannumeral #2}}.#3} % 1: title, 2: book, 3: location
%%% Language Formatting

% Language Input and Display
\usepackage[french, spanish, greek, german, english]{babel}
   %\languageattribute{greek}{ancient}
   \babeltags{en = english}
   \babeltags{de = german}
   \babeltags{el = greek}
   \babeltags{fr = french}
   \babeltags{es = spanish}
   
   \newcommand{\english}[1]{\texten{\textit{#1}}}
   \newcommand{\greek}[1]{\textel{\textit{#1}}}
   \newcommand{\german}[1]{\textde{\textit{#1}}}
   \newcommand{\french}[1]{\textfr{\textit{#1}}}
   \newcommand{\spanish}[1]{\textes{\textit{#1}}}
   
\usepackage{teubner}

\usepackage{ellipsis} % MOVED HERE FROM AFTER TURABIAN

%% Linguistic Characters

% Classical Characters:
% Digamma is pre-coded into teubner as \F\f or \Digamma\textdigamma
% Heta is pre-coded as \h
% Jod/jota is pre-coded as \j
% Semivowel marker for ι/υ is available \c or \semi
% Carons for palatalized consonants are encoded as \cap

% IPA:
\usepackage[extra]{tipa}
    \newcommand{\ipa}[1]{\textipa{#1}}
    
    \newcommand{\hellenic}[1]{\textit{\ipa{#1}}} % Common Hellenic and other historical dialects
    \newcommand{\pie}[1]{\textit{\ipa{#1}}} % Proto Indo European

% My Own Character Sets:
\newcommand{\w}{\english{\.{c}}} 
\newcommand{\W}{\english{\.{C}}}

\newcommand{\vowel}{\english{\.{v}}}
\newcommand{\Vowel}{\english{\.{V}}}

\newcommand{\ash}{\hellenic{\={\ae}}}
\newcommand{\Ash}{\hellenic{\={\AE}}}

\usepackage{amssymb} % for nicer empty set
    \let\emptyset\varnothing
\usepackage{units} % for nicer fraction
    \let\frac\nicefrac
    
    \newcommand{\ea}{$\nicefrac{\textrm{\greek{ε}}}{\textrm{\greek{α}}}$}
    \newcommand{\eo}{$\nicefrac{\textrm{\greek{ε}}}{\textrm{\greek{ο}}}$}
    
    \newcommand{\eeoo}{$\nicefrac{\textrm{e}}{\textrm{o}}$}
    
% Greek Roots
\newcommand{\groot}[1]{$\sqrt{\textrm{\greek{#1}}}$}
\newcommand{\gstem}[1]{$\sqrt[\textrm{stem}]{\textrm{\greek{#1}}}$}

\newcommand{\lroot}[1]{$\sqrt{\textrm{\english{#1}}}$}
\newcommand{\lstem}[1]{$\sqrt[\textrm{stem}]{\english{#1}}$}

% Incorrect Forms:
	%Lets you choose Greek or Latin characters
		% Hopefully
\newcommand{\iform}[2][G]{%
  \ifx G#1 \bgroup\sffamily\english{\textsuperscript{\tiny x}}\egroup\greek{#2}\else
  \ifx L#1 \bgroup\sffamily\english{\textsuperscript{\tiny x}}\egroup\english{#2}%
  \fi\fi
}
			% Well hot damn it works
    
% Various Brackets
\usepackage{textcomp}
    \newcommand{\ortho}[1]{\texten{\textlangle}#1\texten{\textrangle}} % orthographic notation
    \newcommand{\phonet}[1]{[\ipa{#1}]} % phonetic notation
    \newcommand{\phonem}[1]{/\ipa{#1}/}
    
    \newcommand{\spell}[2]{\ortho{\greek{#1}}~\phonem{\ipa{#2}}}
    
%%% Historical Characters
\usepackage{greek6cbc}
\usepackage{greek4cbc}
\usepackage{etruscan}
\usepackage{runic}
\usepackage{linearb}
\usepackage{cypriot}
\usepackage{ugarite}
\usepackage{oldprsn}

%%% Titlepage

\title{Missing Sounds}
\subtitle{Reconstructing Phonemes with Dialectical Analysis}
\author{Thomas Broadwater}
\date{December 30, 2020}
%%% Figures

% Figures
\usepackage{phonrule}
\usepackage{tikz}
    \usepackage{tikz-qtree}
    \usetikzlibrary{decorations.pathreplacing}
\usepackage{amsmath}

% Figure Captions
\AtBeginDocument{%
    \renewcommand{\figurename}{Figura}
    \renewcommand{\thefigure}{\Roman{figure}}
}
%%% Tables

% Tables
\usepackage{longtable}
\usepackage{booktabs}
\usepackage{lscape}
\usepackage{graphicx}
\usepackage{multirow}

% Table Colors
% Found under Text

% Table Captions
\AtBeginDocument{%
    \renewcommand\tablename{Tabula}
    \renewcommand{\thetable}{\Roman{table}}
}
%%% Code Highlighting
\usepackage{minted}
%%% Bibliographies

% a lot is already provided by turabian-formatting
\usepackage[
    autocite = footnote,
    sorting=ynt,
    annotation]
    {biblatex-chicago}
    
    \addbibresource{Bibliographies/Primary}
    
\renewcommand{\bibfont}{\normalsize}
    
\usepackage{csquotes}

% Links
\usepackage{hyperref}
    \hypersetup{
        colorlinks=true,
        linkcolor=blue,
        filecolor=magenta,      
        urlcolor=cyan,
    }
%%% General Editing

% Counter for Draft Paragraphs and Margin Notes
\newcounter{dpara}[section]
\newcommand{\dpara}[1]{\stepcounter{dpara}\label{Par:\thesection:\thedpara}\paragraph{\textcolor{red}{\bgroup\sffamily#1\egroup}}}
\newcommand{\dnote}[1]{\marginnote{\textcolor{red}{\bgroup\sffamily¶\thesection.\thedpara\\ - #1\egroup}}}

% Margin Notes
\usepackage{marginnote}
    \reversemarginpar

% Missing Paragraph
\usepackage{lipsum}
    \newcommand{\missingpara}[1]{\dpara{#1}\dnote{Add #1}\textcolor{gray}{\bgroup\sffamily\lipsum[1]\egroup}}
    
% Things to be Edited
\newcommand{\edit}[2]{\textcolor{red}{#1 [\bgroup\sffamily#2\egroup]}}

% Cross Stuff Out
\usepackage{ulem}
	\newcommand{\rxout}[1]{\textcolor{red}{\xout{\textcolor{black}{#1}}}}
\newcommand{\CH}{Common Hellenic}

\begin{document}

\maketitle
%\tableofcontents\clearpage
%\listofillustrations\clearpage

\section{Homeric}\label{sec:Homeric}
\missingpara{Intro}

\subsection{Elision}\label{subsec:Elision}
\dpara{Elision}
\edit{\ldots}{Transition: something about ``historic scansion.''}
Frequently, that takes the form
of the vowels failing to elide across word boundaries, which may indicate some now-absent
phoneme, present during the composition of the version of the Epics transmitted by our
scribal tradition, but lost some time before the transcription and subsequent editing that gave
us the po\"ems in their modern form. \bookref{Iliad}{15}{214} gives the example
\textel{``Ἡφαίστοιο ἄνακτος''},\autocite[XV.214]{Iliad_1999}
without any contraction between the genitive ending
\textel{-οιο} and the following alpha.
This indicates that the missing sound was a consonant (hereafter \w) and not
a vowel (hereafter \vowel), since a missing vowel would have still allowed elision,
giving the sequence \iform[G]{Ἡφαίστοιο \vowelάνακτος} $\to$ \iform[G]{Ἡφαίστοι' \vowelάνακτος}
$\to$ \iform[G]{Ἡφαίστοι' ἄνακτος}.

\textcolor{red}{\w\ features 1}

\subsection{Meter}\label{subsec:Meter}
\dpara{Scan 1}
\edit{\ldots}{Something about inconsistent representation due to later editing}
Another feature of this historic scansion involves vowels
that are phonologically short scanning as though they were long.
\bookref{Iliad}{5}{308} shows the segment \textel{``ἀπὸ ῥίνον''} scans with a long omicron
\ortho{\textel{\M{ο}}},\autocite[V.306]{Iliad_1999}
Even though in typical orthography, the omicron \ortho{\textel{ο}} exclusively encodes the
short vowel \phonem{o}.
However, standard metrical practice allowed a short vowel to scan as long, with the title
\textit{long by position},
when followed by two or more consonants, whether in the same word or across word boundaries. This implies
that the rho \ortho{\textel{ρ}} in \textel{ῥίνον} was originally part of a consonant
cluster, either \textel{\wρ} or \textel{ρ\w}. A cluster in this position would cause the
preceding omicron \ortho{\textel{ο}} to scan as long, and explain the apparent discrepancy
in the historic scansion.

\dpara{Scan 2}
Outstanding long vowels also point towards the same process occurring word-initially.
Scansion of \bookref{Iliad}{1}{33} shows the word \textel{\Asm{ε}δεισεν} with a short
epsilon \ortho{\textel{ε}} which, as with \textel{ἀπὸ}, scans as long by
position.\autocite[I.33]{Iliad_1999}
This indicates that the root-initial delta \ortho{\textel{δ}} was similarly part of a consonant cluster,
either \textel{\wδ} or \textel{δ\w}, which was able to affect scansion word-medially.
\edit{\ldots}{Something about quantitative alternation}

\subsection{Intent}\label{subsec:Intent}
\missingpara{Intent}
\section{Lengthening}\label{sec:Lengthening}

\subsection{Plausibility}\label{subsec:Plausibility}
\dpara{QA}
Quantitative alternations between Attic and Ionic point to the presence of the historical
consonant \w. \bookref{Iliad}{1}{474} shows \textel{κᾱλός} with a long alpha \ortho{\textel{ᾱ}} where
Attic has a short alpha \ortho{\textel{ᾰ}}. This could be another case of a short vowel
becoming long by position,
but \edit{\ldots}{something about lots of other vowels being long too}. Forms like
\textel{κόρη}, \textel{ξένος}, and \textel{ὅρος} all have Ionic counterparts with a long vowel:
\textel{κούρη}, \textel{ξεῖνος}, and \textel{οὖρος} respectively. So while root-initial \w-clusters caused
a preceding short vowel to scan as long across root boundaries, root-medial clusters caused a
short vowel to become long by nature in Ionic, while having no affect in Attic.
In the cases of epsilon \ortho{\textel{ε}} and omicron \ortho{\textel{ο}}, there is no reason to
assume that the vowels are long by position
as that metrical tactic is never marked with a change in a vowel's spelling, only by the
presence of multiple following consonants.

\subsection{S-Elision}\label{subsec:SElision}
\dpara{S-E 1}
A possible explanation for these long vowels is s-elision, triggering a process called
compensatory lengthening. Terms such as \textel{εἰμί} $\gets$ \CH\ \ipa{*esm\'\i} and \textel{σελήνη} $\gets$
CH \ipa{*sel\'{\u{a}}sn\=a} show that an original *s was elided, which triggered the lengthening of
the preceding vowel.\footnote{Strictly speaking, this is not accurate. The \IE\ *s had become an *h in
in most positions at the beginning of the \CH\ era. However, since the process starts with an IE *s,
and since the particular steps leading towards complete elision are not entirely relevant to this paper,
I have opted to not mark the \CH\ *h. I may come to regret this at some point, but here we are.}
This happens through a process called Mora Preservation, whereby the individual mor\ae,
the relative units of metrical length, are moved in order to maintain the specific length of a word
after an elision.

\textcolor{red}{[Diagram of Mora Preservation]}

\dpara{S-E 2}
S-elision, however, does not provide a satisfactory explanation in the case of \textel{καλός}.
This process occurred late in, or shortly after, the \CH\ period, as evidenced by the differing
reflexes across the Greek dialects; and the \AI\ family actively participated in
compensatory lengthening after s-elision. CH \iform[L]{kals\'os} would have resulted
in \edit{AI \iform[G]{κᾱλός}, with a long alpha in both dialects.}{Vowel Shift after CL would have rendered \iform[G]{κηλός}}\autocite[229]{Sihler_1995}

\subsection{J-Elision}\label{subsec:JElision}
\dpara{J-E 1}
Another explanation may be some lengthening in response to \textel{\yod}-elision. Terms like
\textel{τείνω} $\gets$ CH *t\'e\v{n}\v{n}\=o $\gets$ IE *t\'enjoh\textsubscript{2},\autocite[64; for the endings 
of IE second conjugation thematic verbs.]{mallory_adams_2009}
from a group of
verbs with the so-called `iota present',\footnote{that is, the \IE\ present with the infix -\textel{\yod}\eo-.}
shows a long vowel \phonem{e:} \ortho{\textel{ει}} where the root \groot{τεν-} has a short \phonem{e}
\ortho{\textel{ε}}. This could indicate that the elision of \textel{\yod}\ lead to another
compensatory lengthening.

\dpara{J-E 2}
This is, however, impossible for a few reasons. First, the IE form -\textel{\yod}\eo- created
numerous verb forms in Greek, such as verbs in \textel{-έω}, which typically contracts to \textel{-ῶ}
in Attic, and which shows a short epsilon \ortho{\textel{ε}} where compensatory lengthening
would have created a long vowel \phonet{e:} \ortho{\textel{ει}}. Second, the verb has
an alternate reduplicated form \textel{τιτάινω} which, while agreeing with the root \groot{τ\eaν-},
shows a diphthong \phonet{aj} \ortho{αι} where compensatory lengthening would have created a long
monophthong \iform[L]{\phonet{a:}} \iform[G]{\ortho{ᾱ}}. This suggests that an intervocallic \textel{\yod}\ 
was lost
without triggering compensatory lengthening, and that the long \phonet{e:} \ortho{ει} in \textel{τείνω}
was the reflex of a genuine diphthong, created by the same metathetic process which made the diphthong
in \textel{τιταίνω}. This process was part of a greater system called the Greek
Palatalization,\autocite[197]{Sihler_1995} which, crucially for \textel{κ\B{\M{α}}λός}, changed the segment
\textel{*λ\yod} into \textel{λλ}, as in \textel{ἄλλος} $\gets$ CH *\'al\textel{\yod}os. The result of
\CH\ \iform[L]{k\u{a}l\textel{\yod}\'os} would have been \iform[G]{κᾰλλός}, with a short alpha
\ortho{\textel{ᾰ}} and a geminate lambda \ortho{\textel{λλ}}.

% Table of relevant, non-spoiler palatalization, please!

\subsection{\W-Elision}\label{subsec:WElision}
\dpara{\W-E}
The remaining option is compensatory lengthening induced by \w-elision. Terms like \textel{κόρη},
\textel{μόνος}, and \textel{ξένος} all show alternations in situations where a *Cs or *C\textel{\yod} segment
would have caused lengthening or a diphthong. Since the only other phonemes to elide without conditining
were *s and *\textel{\yod}, the last remaining option is the consonant \w, which was already shown to have
elided in all positions, as indicated by historical scansion. This has two major implications: first,
precedent implies that the sound was some sort of lenis consonant, either a fricative or approximant;
second, the differences between Attic and Ionic reflexes indicate that the consonant \w\ was lost
after the two dialects had split, at a much later date than s- and \textel{\yod}-elision.
\edit{Notably, the elision must have occurred after the \AI\ vowel shift, which can be used to
establish a relative chronology of sound changes.}{This can be incorporated into the following paragraph.}

\textcolor{red}{\w\ features 2}

% Need to distinguish between VwC/V and VCw
\section{R\"uckverwandlung}\label{sec:Ruck}

\subsection{\rxout{Plausibility}}\label{subsec:Plaus2}

\subsection{Chronology}\label{subsec:Chronology}
\dpara{R\"uck 1}\label{para:Ruck1}
The beginning of \w-elision can also be confidently placed after
the onset of the Attic R\"uckverwandlung, which saw the newly-created
vowel \ipa{\ae:} revert back to a long \ipa{/a:/} before a rho \ortho{\textel{ρ}},
epsilon \ortho{\textel{ε}}, or iota \ortho{\textel{ι}}; this is the process which
created the final segment \textel{-έᾱ} in \textel{νέᾱ}. \textel{Κόρη}, however, seems to
violate this rule with its final segment \textel{-ρη}. Given that the AI
quantitative alternation implies the CH root \lroot{kor\w-}, with a root-final
consonant \w\ between the rho \ortho{\textel{ρ}} and what was, at the time,
the feminine ending *-\={a}. If the consonant \w\ had been elided before the AI
vowel raising, the sequence would have appeared as CH \iform[L]{k\'or\w-\=a} $\to$
\iform[L]{k\'or-\=a} $\to$ \iform[L]{k\'or-\ash} $\to$ \iform[G]{κόρᾱ}. Since the consonant \w\ blocked the
process of R\"uckverwandlung, the correct relative sequence can be
confidently reconstructed as *kor\w-\=a $\to$ *kor\w-\ash\ $\to$ kor\w-\ipa{\=E}
$\to$ \textel{κόρη} 

\dpara{R\"uck 2}\label{para:Ruck2}
R\"uckverwandlung, however, could not possibly have happened in one
singular instance. \textel{Νέᾱ}, as mentioned above, still underwent
R\"uckverwandlung, despite the intervocalic hiatus indicating the
original root \groot{νέ\w-}. \W-elision, then, must have elided some time
after R\"uckverwandlung following a rho \ortho{\textel{ρ}}, but before R\"uckverwandlung
following the front vowels epsilon \ortho{\textel{ε}} and iota \ortho{\textel{ι}},
giving the process two distinct phases.

\dpara{Contr.}\label{para:Contr}
Both of these processes occurred after the onset of vowel contractions.
\textel{Νέᾱ}, again as mentioned above, shows intervocalic hiatus in the segment
\textel{-έᾱ}, which is consistent with elision \w-elision in other lemmas.
S- and \textel{\yod}-elision, however, both allow contractions between vowels, as seen in
in the genitive singular \textel{γένους} $\gets$ *g\'enesos,\autocite[172]{Sihler_1995}
and the first person singular \textel{νικῶ} $\gets$ nik\'aj\=o.
If contractions had occurred after \w-elision, terms like \textel{ἐννέα} $\gets$ CH *enn\'e\w\textel{\shwa}
would read \iform[G]{ἐννῆ}
\section{Reconstructions}\label{sec:Recon}

\subsection{Roots}\label{subsec:Roots}
The main process used thus far to find the consonant *\w\ between vowels,
that is, searching for and analyzing intervocalic hiatus, can be
partially automated. For this purpose, I have written three short scripts
in a programming language called R, which help to gather and parse
appropriate evidence.

The first script, called HomeR, is a text miner: a program which reads a
corpus of text, manipulates it, and provides the requested output. At
the present moment, HomeR is directed to read every book of the Iliad,
track each line number (both relative to the book number and to the piece
as a whole), then strip every accent it finds aside from the diaereses.
It then takes every word from its line and places it in a new row of
a table, along with the relevant line and book numbers. In this way, every
single word of the Iliad -- all 111,861 of them, in the edition used here --
is prepared for analysis.

HomeR has two companions, the first of which is named ThRax, for the
Hellenistic grammarian Dionysios Thrax, or \textel{Διονύσιος ὁ Θρᾷξ}.
ThRax is a Sound Change Applier (SCA), whose sole function is to receive
a hypothetical reconstructed form of a word, and estimate a descendant
form. This relies on the premise implied by the relative chronology
above, that sound changes happen in a rigid order. This means that sound
changes can be described in a program and systematically applied, allowing
a reconstructed word form to be passed through a list of changes and
compared to the attested word order in order to test the veracity of the
reconstruction.

The second companion is named GoRgias, after the ancient sophist. GoRgias
is responsible for filtering the output from HomeR in order to build a table of
potentially useful terms, then guessing at a hypothetical reconstruction,
which it passes along to ThRax. After the reconstructions are run through
the relevant sound changes, it and its hypothetical descendant are added
to the table so that forms can be compared.

These three scripts have been directed to read through the Iliad, then
filter the resulting table down to one particular scene, Hector's farewell
to Andromache, specifically \bookref{IL}{6}{390-412, 466-482}. % Break para to avoid back-to-back fig.s

\textcolor{red}{[Add the farewell scene]}

This passage
is then filtered again to find any instances of intervocalic hiatus, which
are then separated by inserting the consonant *\w. This happens in three
phases. First, the consonant is placed between two monophthongs, such as
\ipa{/e.o/} or \ipa{/a.E:/}. Second, it is placed between a diphthong and
a monophthong, such as \ipa{/a\textsubarch{I}.E:/} or \ipa{/y\textsubarch{I}.o/}.
The instructions to GoRgias are worded in such a way that this step will also
capture hiatus between diphthongs, such as \ipa{/a\textsubarch{I}.o\textsubarch{I}/}.
Lastly, the consonant is added between a monophthong and a diphthong, such
as \ipa{/e.o\textsubarch{I}/}.

Running GoRgias with these parameters provides a small number of roots
which may have once held the consonant *\w.

\textcolor{red}{[Example of an output from GoRgias]}

It should be noted that these scripts are somewhat greedy, and will regularly
over-capture. For instance, the denominative suffixes \textel{-ιος} $\gets$
\IE\ *-ios, \textel{-ιᾰ} $\gets$ IE *-i\textsubring{h}\textsubscript{2}, and
\textel{-ίη} $\gets$ IE *-ieh\textsubscript{2} are all regularly captured and
reconstructed as \CH\ \iform[L]{-i\w os}, \iform[L]{-i\w a}, and \iform[L]{-i\w\=a}.
This is, however, currently preferable to under-capturing, as some word forms
would be inappropriately excluded, such as \textel{ἰάχω} and \textel{Ἀχαιοί}
$\gets$ CH *\w i\w ak\textsuperscript{h}\=o and
*Ak\textsuperscript{h}a\textsubarch{\textsci}o\textsubarch{\textsci}. Following
through to confirm captured words returns a list of ten roots in the passage
which may have
had the consonant \w\ at the end of the Common-Attic-Ionic period.

\textcolor{red}{[Roots Table]}
%\begin{table}[htbp]
\centering
\begin{tabular}{@{}lllllll@{}}
\toprule
Attested &
  Reconstructed &
  PAI Root &
   &
  Attested &
  Reconstructed &
  PAI Root \\ \midrule
\greek{Σκαιάς} &
  *\hellenic{ska\textsubarch{I}\.{c}as} &
  *\lroot{\hellenic{ska\textsubarch{I}\.{c}-}} &
   &
  \greek{πάϊς} &
  *\hellenic{pa\.{c}is} &
  *\lroot{\hellenic{pa\.{c}-}} \\
\greek{θέουσα} &
  *\hellenic{t\super{h}e\.{c}o:sa} &
  *\lroot{\hellenic{t\super{h}e\.{c}-}}  &
   &
  \greek{ἰάχων} &
  *\hellenic{\.{c}i\.{c}ak\super{h}O:n} &
  *\lroot{\hellenic{\.{c}ak-}} \\
 %%%%%%%%%%%%%%%%%%%
\greek{χέουσα} &
  *\hellenic{k\super{h}e\.{c}o:sa} &
  *\lroot{\hellenic{k\super{h}e\.{c}-}} &
 %%%%%%%%%%%%%%%%%
   &
  \greek{υἱόν} &
  *\hellenic{hy\textsubarch{I}\.{c}on} &
  *\lroot{\hellenic{hy\textsubarch{I}\.{c}-}} \\
\greek{ἐλεαίρες} &
   *\hellenic{ele\.{c}a\textsubarch{I}res} &
   *\lroot{\hellenic{ele\.{c}-}} &
   &
  \greek{ἐρύετο} &
  *\hellenic{\.{c}eryeto} &
  *\lroot{\hellenic{\.{c}eru-}} \\
\greek{Ἀχαιοί} &
  *\hellenic{ak\super{h}a\textsubarch{I}\.{c}o\textsubarch{I}} &
  *\lroot{\hellenic{ak\super{h}a\textsubarch{I}\.{c}-}} &
   &
   &%\greek{θεοὶ} &
   &%*\hellenic{t\super{h}e\.{c}o\textsubarch{I}} &
   \\ \bottomrule
\end{tabular}
\caption[Reconstructed Common-Attic-Ionic Roots]{Reconstructed Common-Attic-Ionic Roots}
\label{tab:recon-roots}
\end{table}

\subsection{Suffixes}\label{subsec:Suffixes}
Together, the scripts also identify a number of suffixes with the consonant *\w.
This includes some more common suffixes, such as \textel{-αρ, -ων}, and,
in the passage specifically, \textel{-εις}.
Some, such as \textel{-ος}, can be confidently distinguished from similar forms --
here, the o-stem nominative \textel{-ος} -- by their extraneous hiatus,
but others require more investigation.

The suffixes \textel{-αρ} and \textel{-ων}, found in \textel{στέαρ} and \textel{πίων},
are both confirmed by examining other derivations from shared roots. \textel{Στέαρ}
shares a root with both \textel{στάσις} and \textel{ἵστημι}, which provide
the root \groot{στ\ea-}, with the alternations in vowel quality being a result
of IE ablaut; \footnote{Cf. Smyth 35-56} while \textel{Πίων} shares its root with
\textel{πιμέλη}. Given that the forms \textel{στέαρ} and \textel{πίων} show signs
of a missing consonant, while their cognate derivatives do not, it can be
confidently argued that the consonant was a part of the suffix, not the roots.
Both forms disqualify an intervocalic *-s- or *-j-, given the lack of any
discernible effect on the preceding vowel following elision.
No compensatory lengthening occurred, disqualifying s-elision, and no
diphthongs were formed, disqualifying j-elision. The remaining options, then,
are the forms \textel{στέαρ} $\gets$ CAI *st$\nicefrac{e}{a}$-\w ar and
*pi-\w \=on. 

The base form of *-\w\=on, however, must have had a short vowel.
The suffix shows quantitative alternation between cases, with the
Genitive Singular \textel{-ονος} and the Dative Singular \textel{-ονι},
suggesting that the omicron \spell{\textel{ο}} is lengthened when in the
word-final position in much the same way as the final vowel in
\textel{πάτηρ} and \textel{ἄνηρ}. This gives the suffix' base form
\textel{-on-}, reconstructed as PAI *-\w\u{o}n-.

\textcolor{red}{\textel{-ων} endings here}

The suffix \textel{-εις} can be reconstructed with an initial *\w.
It is attested three times in the passage, in the words \textel{ὑληέσσῃ},
\textel{Τρώεσσιν}, and \textel{βροτόεντα}.
It can be reasonably expected that the suffix begin with a consonant
given the construction of the word \textel{βροτ-ό-εντα}, attested with
an omicron \spell{\textel{ο}} inserted as a connecting vowel, which only
occurred when the addition of an affix to a stem would have created a
consonant cluster. The omicron, then, must have been added to separate
the root-final tau \spell{\textel{τ}} and whatever consonant followed.
If that consonant had been *-s- or *-j-, then its elision would have
allowed for contraction between the connecting omicron \spell{\textel{ο}}{o}
and the suffix-initial epsilon \spell{\textel{ε}} giving the form
\iform[G]{βροῦντα}. The initial consonant, then, must have again been the
hypothetical consonant \w.

Much like the suffix \textel{-ων}, the Nominative forms of \textel{-εις}
obfuscate the base form. Looking at the Genitive, Dative, and Accusative
forms -- \textel{-εντος}, \textel{-εντι}, and \textel{-εντα}, respectively --
show the base form \textel{-εντ-}. The Nominative form can be explained
as the result of two successive consonant contractions, giving the
process \textel{-εντς} $\to$ \textel{-εινς} $\to$ \textel{-εις}, with
the long epsilon \spell{\textel{ει}}{e:}
arising through compensatory
lengthening. This indicates the CAI form *-\w ents.

\dpara{(e)u} GoRgias also demonstrates a connection between the two
u-type agent suffixes, \textel{-υς} and \textel{-ευς}. The script, when
instructed to examine the chosen passage, captures \textel{υἱόν}, the
Accusative Singular of \textel{υἱύς}; when instructed to examine the
Iliad as a whole, it captures such terms as \textel{ἡδύς},
\textel{Ἀχιλλεύς}, and \textel{βασιλεύς}. The declension of the -type
endings show a strong concordance in their endings (though
their exact form varies somewhat to satisfy metrical requirements).
This indicates an etymological relationship between the two, where
\textel{-ευς}, with a present epsilon \spell{\textel{ε}}{e}, is the
thematic form of \textel{-υς}, which lacks any such vowel. 

\textcolor{red}{\textel{-(ε)υς} endings here}

\dpara{BRING IT HOME}
These endings are also noteworthy for what they indicate about the
consonant *\w: the position of the intervocalic hiatus in these
endings can indicate the specific quality, the actual pronunciation,
of the hypothetical consonant.
Take the Dative Singular ending, variously attested as \textel{-ηϊ}
or \textel{-εϊ}. The hiatus stands exactly where the Nominative
singular shows the upsilon \spell{\textel{υ}}{y}, which apparently
disappeared in declension -- the dative, otherwise, would read as
\iform[G]{-ηυϊ} or \iform[G]{-ευι}. It has already been determined
that the consonant was likely either an approximant or a fricative,
and given that it was related to the upsilon, and that some vowels
have a relative approximant, specifically in the form of a relative
semivowel, the pronunciation of the upsilon
can potentially identify the consonant \w's specific pronunciation.

However, given the historical nature of this relationship, it is not
useful to examine the pronunciation of upsilon \ortho{υ} as it was
pronounced in the classical era. At this point, both Attic and
Ionic speakers pronounced this as the front round vowel \ipa{/y/}
(cf. Fr \ortho{\textrm{u}} and De \ortho{\textrm{\"u}}, however it
was fronted some time between 700 and 400 \textsc{b.c.e.}, moving
from its original pronunciation as a back round vowel \ipa{/u/} (cf.
Fr \ortho{\textrm{ou}} and De \ortho{\textrm{u}}).\autocite[529]{malikouti-drachman_bortone_2015}
In this instance, then, the vowel upsilon \ortho{υ} needs to be
analyzed as that back round vowel. 

\textcolor{red}{[Semivowel relationship chart]}
%\begin{table}[htbp]
\centering
\caption{Greek Phonemes with Relative Semivowels}
\label{tab:semivowels}
\begin{tabular}{lll}
Greek                       & Vowel   & Semivowel \\
\textel{ι}                  & \ipa{i} & \ipa{j}   \\
\multirow{2}{*}{\textel{υ}} & \ipa{y} & \ipa{4}   \\
                            & \ipa{u} &          
\end{tabular}
\end{table}

With this in mind, the behavior of the upsilon \ortho{υ} indicates
that the historic consonant likely was pronounced as \ipa{/w/},
the sound of the Gnglish letter double-u \ortho{\textrm{w}}. 
The pronunciation of the two sounds are near-identical, excepting
for differences in quantity -- that is, the vowel \ipa{/u/} is typically
held longer than the semivowel \ipa{/w/}. In fact, the specific features
of these sounds -- literally the physical methods used in their pronunciation,
such has tongue position and tension -- have a near perfect correlation,
except that \ipa{/w/} is labeled as a consonant and \ipa{/u/} a vowel.
This would explain the behavior of the upsilon \ortho{υ} in u-type
endings, with the short, intervocalic /u/ being re-bracketed, being
understood not as the end of the prior syllable but the beginning of the
latter, and then re-analyzed as a semivowel, which was subsequently elided.
The dative, then, progressed from the Common-Attic-Ionic
*-$\nicefrac{e}{\bar{e}}$u.i $\to$ *-$\nicefrac{e}{\bar{e}}$.ui $\to$
*-$\nicefrac{e}{\bar{e}}$.wi $\to$ Homeric
-\textel{$\nicefrac{\textrm{\textel{ε}}}{\textrm{\textel{η}}}$ϊ}

\textcolor{red}{[Input final reconstruction!!!?!?!1`!]}

\clearpage
\printbibheading
\printbibliography[heading=subbibliography,title={Cited Texts}]

%\clearpage
%\section*{Appendix}

%\subsection*{HomeR.R}
%\inputminted{R}{HomeR/HomeR.R}

%\subsection*{ThRax.R}

%\subsection*{GoRgias.R}
%\inputminted{R}{HomeR/GoRgias.R}

\tableofcontents\clearpage
\listofillustrations

\clearpage
\section{Cold Open}\label{sec:ColdOpen}

When we try to reconstruct Ancient Greek, we are wrong.  Reconstructive linguistics, like any other scientific process, is one of trial and error.  Philologists are constrained by history's slow erosion of knowledge: writing appearing just too late to capture a sound change, delicate scrolls being lost to unforgiving elements, humans destroying that which does not seem worth saving.  Our viewpoint of how people spoke and wrote in prehistoric times has been whittled away by every lost text into the smallest of apertures.  Our only means of understanding the past is oblique by necessity.  It is the process of looking at the holes left by historical changes and constructing the pieces that may have once fit there.  There are many such holes in our understanding of Ancient Greek, and one that fits well enough for me.

To be specific, what is missing is the quality of a phoneme: its actual specific pronunciation. There are, broadly speaking, two ways of describing a phoneme's quality: first by specifying its place and manner of articulation relative to the mouth, and second by identifying its distinctive features.

In consonants, the place of articulation describes where the tongue rests in the mouth during an utterance, such as between the teeth (``interdental'') or against the soft palate (``velar''). The manner of articulation describes the quality of airflow through the mouth during utterance. A stop, such as \ipa{/t/}, allows no airflow, while a fricative, such as \ipa{/f/}, allows a highly-turbulent flow.

Place is similar in vowels, but since they are more sonorant (that is, they create less turbulence in the vocal tract), they are described in terms of how high and front the tongue is during utterance. The vowel \ipa{/i/} is called a high front vowel, since the tongue is close to the roof of the mouth and extended towards the teeth, while \ipa{/A/} is a low back vowel, since the tongue is against the bottom of the mouth, and retracted towards the pharynx.

\begin{table}[htbp]
\rowcolors{2}{gray!25}{white}
\centering
\caption{Reconstructed Attic-Ionic Phonemic Consonants.}
\label{tab:Greek-Cons}
\resizebox{\textwidth}{!}{%
\begin{tabular}{@{}rcccccccccc@{}}
\toprule
 &
  \multicolumn{2}{c}{Labial} &
  Dental &
  \multicolumn{2}{c}{Alveolar} &
  Post-Alveolar &
  Palatal &
  \multicolumn{2}{c}{Velar} &
  Glottal \\ \midrule
Stop &
  \ipa{/p/} \greek{p} &
  \ipa{/b/} \greek{b} &
  \multicolumn{2}{c}{\ipa{/t/} \greek{t}} &
  \multicolumn{2}{c}{\ipa{/d/} \greek{d}} &
   &
  \ipa{/k/} \greek{k} &
  \ipa{/g/} \greek{g} &
   \\
Aspirated Stop &
  \ipa{/p\super{h}/} \greek{f} &
   &
  \multicolumn{2}{c}{\ipa{/t\super{h}/} \greek{j}} &
   &
   &
   &
  \ipa{/k\super{h}/} \greek{q} &
   &
   \\
Nasal &
   &
  \ipa{/m/} \greek{m} &
   &
   &
  \multicolumn{2}{c}{\ipa{/n/} \greek{n}} &
   &
   &
  \ipa{/N/} \greek{g($\nicefrac{\textrm{\greek{g}}}{\textrm{\greek{k}}}$)} &
   \\
Trill &
   &
   &
   &
   &
  \multicolumn{2}{c}{\ipa{/r/} \greek{r}} &
   &
   &
   &
   \\
Fricative &
   &
   &
   &
  \ipa{/\textsubbar{s}/} \greek{sv} &
  \ipa{/\textsubbar{z}(d)/} \greek{z} &
   &
   &
   &
   &
  \ipa{/h/} \greek{\<{v}} \\
Approximant &
   &
   &
   &
   &
   &
   &
  \textcolor{gray}{\ipa{/*j/} \greek{\j}} &
   &
   &
   \\
Lateral Approximant &
   &
   &
   &
   &
  \multicolumn{2}{c}{\ipa{/l/} \greek{l}} &
   &
   &
   &
   \\ \bottomrule
\end{tabular}%
}
\end{table}
\begin{table}[htbp]
\rowcolors{2}{gray!25}{white}
\centering
\resizebox{\textwidth}{!}{%
\begin{tabular}{@{}rrlrlrlccccrlrlcrrlrlrlrlccccrlrlrl@{}}
\toprule
\multicolumn{1}{l}{} &
  \multicolumn{14}{c}{Short Vowels} &
  \multicolumn{1}{l}{} &
  \multicolumn{1}{l}{} &
  \multicolumn{18}{c}{Long Vowels} \\ \midrule
 &
  \multicolumn{2}{c}{Front} &
  \multicolumn{1}{c}{} &
  \multicolumn{1}{c}{} &
  \multicolumn{1}{c}{} &
  \multicolumn{1}{c}{} &
   &
  \multicolumn{2}{c}{Central} &
   &
  \multicolumn{1}{c}{} &
  \multicolumn{1}{c}{} &
  \multicolumn{2}{c}{Back} &
   &
   &
  \multicolumn{2}{c}{Front} &
  \multicolumn{1}{c}{} &
  \multicolumn{1}{c}{} &
  \multicolumn{1}{c}{} &
  \multicolumn{1}{c}{} &
  \multicolumn{1}{c}{} &
  \multicolumn{1}{c}{} &
   &
  \multicolumn{2}{c}{Central} &
   &
  \multicolumn{1}{c}{} &
  \multicolumn{1}{c}{} &
  \multicolumn{1}{c}{} &
  \multicolumn{1}{c}{} &
  \multicolumn{2}{c}{Back} \\ \cmidrule(lr){2-15} \cmidrule(l){18-35} 
High &
  \ipa{i} &
  \greek{ῐ} &
   &
   &
   &
   &
   &
   &
   &
   &
   &
   &
  \ipa{u} &
  \greek{ῠ} &
   &
  High &
  \ipa{i:} &
  \greek{ῑ} &
   &
   &
   &
   &
   &
   &
   &
   &
   &
   &
   &
   &
   &
   &
  \ipa{u:} &
  \greek{ῡ} \\
Mid-High &
   &
   &
  \ipa{e} &
  \greek{e} &
   &
   &
   &
   &
   &
   &
  \ipa{o} &
  \greek{o} &
   &
   &
   &
  Mid-High &
   &
   &
  \ipa{e:} &
  \greek{ει} &
   &
   &
   &
   &
   &
   &
   &
   &
   &
   &
  \ipa{o:} &
  \greek{ου} &
   &
   \\
Mid-Low &
   &
   &
   &
   &
   &
   &
   &
   &
   &
   &
   &
   &
   &
   &
   &
  Mid-Low &
   &
   &
   &
   &
  \ipa{E:} &
  \greek{η} &
   &
   &
   &
   &
   &
   &
  \ipa{O:} &
  \greek{ω} &
   &
   &
   &
   \\
Low &
   &
   &
   &
   &
  \ipa{a} &
  \greek{ᾰ} &
   &
   &
   &
   &
   &
   &
   &
   &
   &
  Low &
   &
   &
   &
   &
   &
   &
  \ipa{a:} &
  \greek{ᾱ} &
   &
   &
   &
   &
   &
   &
   &
   &
   &
   \\ \bottomrule
\end{tabular}%
}
\caption{Attic-Ionic Phonemic Vowels -- Before Attic shifted \ipa{/u(:)/} to \ipa{/y(:)}}
\label{tab:Greek-Vowels}
\end{table}

Distinctive features are individual components of pronunciation, starting as vague as whether a phoneme is either a consonant [+consonant] or vowel [+vowel], and ending as specific as whether a phoneme is pronounced by obstructing airflow with the sides of the tongue [+laminal] or the tip [+apical]. Features can be written in-line, as shown here, or in a matrix, which is the typical method when writing formal sound change rules.
\section{Homeric}\label{sec:Homeric}
\dpara{Intro} The Iliad of Homer never had a standard form, rather it varied between poets for the audience in attendance. Instead of learning the entire piece, these poets memorized prebuilt lines, which they used as a foundation to improvise the rest of the epic. These are the oldest lines of the Iliad, surviving from well before the first scribes recorded the poem. As such, they show irregularities stemming from historic scansion of static phrases.

\dpara{Apocope} Such irregularities often take the form of the vowels failing to elide across word boundaries. \bookref{Iliad}{15}{214} gives the example \greek{Ηφαίστοιο ἄνακτος},\autocite[XV.214]{Iliad_1999} without any apocope, or vowel elision, between the genitive ending \greek{-οιο} and the following alpha. In Greek, like in French and Italian, a word-terminal vowel will often elide before a word-initial vowel. That is to say, a vowel at the end of a word will drop if the following word begins with another vowel. Compare the elision in Greek  \greek{μεθ' ἡμῖν} ``with us'', from \greek{μετὰ ἡμῖν}, to French  \french{J’ai sommeil} ``I am tired'' $\gets$ \french{Je ai sommeil}. This implies that there was some phoneme, some distinct sound, was present at the start of the word \greek{ἄνακτος}. This phoneme was a consonant, which will be written as *\w\ until the exact sound can be determined, rather than a vowel, *\vowel, as a vowel would have still allowed elision. The sequence with a vowel would read \iform[G]{Ἡφαίσοιο \vowelάνακτος} to \iform[G]{Ἡφαίστοι' \vowelάνακτος} $\to$ \iform[G]{Ἡφαίστοι' ἄνακτος}. Note that an asterisk before any form means that it is unattested, meaning that it does not survive in the written record. As such, that form is hypothetical, based on the current understanding of Greek. A superscript \textit{x} before a form means that it is expected to be incorrect. Either it is a hypothetical form that does not fit with the current understanding of Greek, or it is a form that contradicts extant records.

\noindent\textcolor{red}{input [+consonant]}

\dpara{LBP}\dnote{2 ¶s?} Another type of historic scansion involves short vowels scanning as long. \bookref{Iliad}{5}{308}  has the phrase \greek{ἀπὸ ῥίνον}, with the omicron \ortho{\textel{ο}} scanning as long, even though the letter only represents a short vowel.\autocite[V.803]{Iliad_1999} This breaks rank with the traditional epic scansion, that is, the means of breaking a line of poetry into a series of short and long syllables. The Iliad, like most epics, uses a system called dactylic hexameter. Each line divides into six units, hence the \textit{hex-} from Greek \greek{ἕξ}\ ``six''. Each unit divides into three syllables -- one long followed by two shorts -- or two syllables -- two longs in a row. These resemble the shape of a finger, \textit{dactyl-} coming from Greek \greek{δάκτυλος} ``finger''. long syllables represent the first bone of the finger, while the short syllables represent the second and third bones. In any dactylic meter, a short vowel followed by one or no consonants creates a short syllable. A short vowel followed by two or more consonants creates a long syllable, regardless of if those consonants are in the same word, or across a word boundary. A long vowel creates a long syllable in any position. In this instance, the omicron \ortho{\textel{ο}} should scan as short, being a short vowel only followed by one consonant. Given that this omicron scans as long \ortho{\textel{ο}}, the following rho \ortho{\textel{ρ}} must be an element of a consonant cluster. The apocopic forms \greek{ἀπ'} and \greek{ἀφ'} disqualify the form \greek{ἀπό\w}, leaving only two possible positions for the hypothetical consonant: either before the rho, \ortho{\textel{*\wρ}}, or after it, \ortho{\textel{*ρ\w}}.

\dpara{LBP II} \dnote{Cut?} Affixes can also point to a similar process happening within a word boundary. Scansion of \bookref{Iliad}{1}{33} shows the word \greek{\Asm{ε}δεισεν},\autocite[I.33]{Iliad_1999} with a word-initial epsilon \greek{ἐ-} counting as a long vowel. In the aorist tense, or the ``simple past'' tense, the verb takes a prefix called the epsilon augment. As with the letter omicron, epsilon encodes a short vowel, implying that another consonant is missing. The possible clusters, then, are \ortho{\textel{*\wδ}} and \ortho{\textel{*δ\w}}.

\section{Lengthening}\label{sec:Lengthening}

\subsection{Identifying CL}
\dpara{QA}\dnote{2 ¶s?} While the presence of the historic consonant \greek{*\w} effected the scansion of historic phrases, its loss introduced discernible effects on a preceding vowel in the same root. Epic Greek shows \greek{κούρη}, \greek{ξείνος}, \greek{οὖρος}, and \greek{κᾱλός} where Attic Greek shows \greek{κόρη}, \greek{ξένος}, \greek{ὅρος}, and \greek{κᾰλός}. These are distinguishable from vowels made long by position given that the Epic vowels show their length with spelling. That is, whereas \greek{ἀπό} and \greek{ἔδεισεν} do not mark length on the omicron or epsilon, \greek{κούρη} and \greek{οὖρος} mark length with a digraph \ortho{\greek{ου}}, and \greek{ξείνος} with \ortho{\greek{ει}}. Note that these digraphs do not represent the sequences \greek{ο + υ} and \greek{ε + ι}, but rather the long vowels \greek{\M{ο}} and \greek{\M{ε}}. The process responsible for these long vowels is called compensatory lengthening. Every phoneme in Greek can be measured in units of duration called mor\ae: short vowels and single consonants are said to have one mora, long vowels and diphthongs have two, and consonant clusters have two or more. When a consonant cluster with \greek{*\w} is reduced, the single mora from \greek{*\w} is transferred to a preceding vowel, as long as the two share a root. This process is significant in identifying the quality of \greek{*\w}, that is, its specific pronunciation. Since there is a known input, a short vowel and a consonant cluster with \greek{*\w},  a known condition, both are in the same root, and a known output, a long vowel and a single consonant, this process is comparable to other known cases of elision in Greek.

\noindent\textcolor{red}{Input Mora Preservation}

\subsection{H-Elision}
\dpara{HE I} The first possible option is H-Elision. As the Proto-Indo-European language developed into the Common-Hellenic language, the consonant \pie{*s} often shifted to the consonant \pie{*h}. This occurred at the beginning of a word, before a vowel (unless following a voiceless stop), after a vowel (unless at the end of a word or followed by a voiceless stop), and in a word-initial consonant cluster with either a nasal or an approximant.\autocite[168-172]{Smyth_2013} \edit{That is, the sequences \pie{*sm}, \pie{*sn}, \pie{*sl}, and \pie{*sr} became \hellenic{*mh}, \hellenic{*nh}, \hellenic{*lh}, and \hellenic{*rh}; for example, PIE \pie{*srud\super{h}mos} became CH \hellenic{*rhut\super{h}mos}, with a voiceless \hellenic{*/\textsubring{r}/}.\autocite[ῥυθμός]{Beekes_2009}}{Necessary?} The consonant \hellenic{*h} disappeared in Greek unless at the beginning of a word and, when part of a consonant cluster, this triggered compensatory lengthening. For example, Common-Hellenic \hellenic{*selahna} became Attic \greek{σελήνη}.\autocite[\textel{σελήνη}]{Beekes_2009}

\dpara{HE II} However, this does not explain the vowel alternations between the Attic and Homeric forms above. Attic and Homeric both agree on vowel lengths in the roots \groot{δην-} in A \greek{δήνεα} and H \greek{δῆνος} $\gets$ CH *\lroot{denh-},\autocite[\textel{δήνεα}]{Beekes_2009} and \groot{σελην-} in A and H \greek{σελήνη} $\gets$ CH *\lroot{selahn-}, as above. if H-Elision were responsible for the long vowels in Homeric \greek{ξείνος} and \greek{κούρη}, then the vowels in Attic would have matched. Furthermore, the vowel qualities are incorrect for compensatory lengthening after H-Elision. Homeric \greek{δῆνος} shows a long eta \ortho{\textel{η}}, cf. French \french{\`e}, while \greek{ξείνος} shows a long epsilon-iota digraph \ortho{\textel{ει}}, cf. F \french{\'e}. This means that H-Elision occurred before the long vowel \hellenic{*\=e} in CH lowered to \greek{η}, while the process which created the long \greek{ει} occurred after.

\subsection{J-Elision}
The last possible option is J-Elision. The semivowel \pie{*j}, cf. English \ortho{y}, itself very common in PIE, elided after another consonant by the end of the Common-Hellenic period.\autocite[196]{Smyth_2013} The loss of this often created long vowels of its own, for example Attic \greek{τείνω} $\gets$ PIE \pie{*t\'enjoh\textsubscript{2}} shows a long \ortho{\greek{-ειν-}} derived from an original  \pie{*-enj-}.\autocite[τείνω]{Beekes_2009} In this respect, the result of J-Elision compares to the result of H-Elision and Compensatory Lengthening.

Yet this analysis is misleading: while the input and output resemble those of H-Elision, the paths taken by the two processes are very different. The first sign is the quality of the long vowel created after J-elision. Given that this process occurred at a similar time to H-Elision, the vowel created by Compensatory Lengthening would have been \iform[G]{η}, as in H \greek{δῆνος} above, giving the form \iform[G]{τήνω}. Worse yet, the root of \greek{τείνω} shows frequent vowel alternations, with attested forms including \greek{τείνω}, \greek{τάνυται},\autocite[τάνυται]{Beekes_2009} and \greek{τιταίνω}.\autocite[τιταίνω]{Beekes_2009} This alternation suggests a different source of the iota \ortho{\textel{ι}} in \greek{τιταίνω}, and by extension \greek{τείνω}, since lengthening of a short \pie{*ᾰ} from PIE \pie{*titanjoh}\textit{\textsubscript{2}} would have resulted in a long \iform[G]{ᾱ}. The culprit here is a sound change called metathesis (G \greek{μετάθεισις}), wherein phonemes or syllables trade places in a word. In English, this is the process responsible for creating \english{aks} as an alternate form  of ask \english{ask}. In Greek, this specific instance of metathesis involved the semivowel \hellenic{*j} trading positions with,  among others, \hellenic{*n}, \hellenic{*r}, and \hellenic{*h}. In the instances of \greek{τείνω} and \greek{τιταίνω}, this change took place through a medial form: Proto-Indo-European \pie{*-\vowel nj-} $\to$ CH \hellenic{*-\vowel\~n\~n-} $\to$ A and H \greek{-\vowelιν-} (CH \hellenic{*\~n} being identical to Spanish \spanish{\~n}).

With all this in mind, it is impossible to assert that \w\ be either \hellenic{*s} or \hellenic{*j}, but these comparisons do provide insight to the quality of \w. During the transition from the Common-Hellenic language to Attic and Homeric, the only three consonants to elide without conditioning were \hellenic{*h}, \hellenic{*j}, and \w. That is to say, any other sound that disappeared from a word did so as the result of some other grammatical or phonological process. For example, A \greek{πᾶς} shows the root \groot{πάντ-}. While the cluster \ortho{\greek{-ντ-}} does elide, it is a conditioned change: dental consonants, and clusters of dentals, elide when they come into contact with a sigma \ortho{\greek{σv}}. In this case, the final sigma \ortho{\greek{-s}} is the nominative ending.  So in Attic, the process is as follows: \greek{πάντ-ς} to \greek{πάν-ς} $\to$ \greek{πᾶ-ς}. Regular H- and J-Elision, on the other hand, are unconditioned changes: \hellenic{*h} and \hellenic{*j} elide in most every case, with only small concessions to the surrounding phonemes. Given that most most elisions involve \hellenic{*h}, and \hellenic{*j}, a comparison to \w\ can be drawn by examining the overlap between these phonemes. Namely, \hellenic{*h} and \hellenic{*j} are not stop consonants: neither involve the complete blocking of airflow through the mouth like \ipa{/p/} and \ipa{/k/}. The phoneme \hellenic{/*h/} is a fricative, pronounced by restricting the glottis (the ``vocal folds'') to create turbulent airflow. The phoneme \hellenic{/*j/} is an approximant, pronounced by creating some turbulence between the tongue and the roof of the mouth, though not enough to qualify as a fricative. While not confirmatory, this gives precedence to analyze the missing consonant \w\ as either a fricative or approximant.

\noindent\textcolor{red}{input [+-fric / +- approx]}
%\section{R\"uckverwandlung}\label{sec:Ruck}
\section{Chronology}\label{sec:Chrono}

\subsection{Elision}

The differing results of Compensatory Lengthening following H- and \W-Elision make it necessary to determine the general order of sound changes between the Common-Hellenic language and the Attic-Ionic dialect family. Chronologies of this sort build on one fundamental premise: sound changes happen in a set order. This need not be the exact same order between languages, but it is stable throughout a language's life. This is because they happen over time, one-by-one, and changing the order would necessitate changing the language as spoken at some point in the past -- something which I cannot do, at any rate. Given this, it is possible to place sound changes into a relative chronology, a list irrespective of the actual date of change. Compensatory Lengthening provides a convenient starting point.

As mentioned above, the eta \ortho{\greek{η}} in H \greek{δῆνος} and the epsilon-iota \ortho{\greek{ει}} in H \greek{ξείνος} imply that H-Elision occurred before \W-Elision. The key here is the height of the long vowels. The short \hellenic{/*e/} (English \english{a} in \english{ate}) in the root \lroot{*denh-} is not only lengthened, but also lowered to a long \hellenic{/*E:/} (E \english{e} in \english{end}). There are three possible explanations to this. First, the short \hellenic{/*e/} could have itself been a short \hellenic{/*E/}. Unfortunately, this would raise the question of why \greek{ξείνος} shows a different result from Compensatory Lengthening. If it were a short \hellenic{/*E/}, then the natural reflex would have been \iform[G]{ξήνος}. Second, the short vowel \hellenic{/*e/} may have naturally evolved to \hellenic{/*E:/} in a single motion when undergoing compensatory lengthening. The issue, again, is that this leaves no room for \greek{ξείνος}: if \hellenic{/*e/} lengthened directly to \hellenic{/*E:/}, then the root \lroot{*ksen\w-} should evolve into \iform[G]{ξήνος}. The third option is that \greek{δῆνος} was the result of a two step process. The short \hellenic{/*e/} in the root \hellenic{*denh-} first became a long \hellenic{/*e:/}, which was then lowered to \hellenic{/E:/} as part of an unconditioned shift. The shift from CH \hellenic{/*e:/} from PIE \pie{*e} or \pie{*eh}\textit{\textsubscript{1}}, to G \hellenic{/E:/} is a known feature. The long eta \ortho{\greek{η}} in the negative particle \greek{μή} is the result of CH \hellenic{*m\=e}, and in \greek{ἡμι-} from CH \hellenic{*s\=emi-}.\autocite[51]{Smyth_2013} The long epsilon-iota in \greek{ξείνος}, then, must have occurred after this general lowering. 

H- and J-Elision, on the other hand, must sit adjacent to one another in the chronology. Attic and Homeric share the majority of reflexes,\autocite[198]{Smyth_2013} for example, PIE \pie{*-gy-} and \pie{*-dy-} become A/H \greek{ζ} as in \greek{ἅζομαι} from \pie{*hag-y\eeoo-} and \greek{ἐλπίζω} from \pie{*elpid-yoh}\textit{\textsubscript{2}}.\autocite[200]{Smyth_2013} The consistent reflexes imply that J-Elision occurred before the appearance of distinctly Attic and Homeric features. Following that mindset, the different reflexes of \W-Elision in Attic and Homeric then imply that, while both dialects lost the hypothetical consonant *\w, they did so after the dialect family split, allowing for differences in derived terms.

\subsection{R\"uckverwandlung}
Dialectical differences also show that \W-Elision occurred before the completion of the Attic-Ionic Vowel Shift, placing it near to the very beginning of the split in the Attic-Ionic family. Where Homeric shows the form \greek{κούρη}, with a long omicron-upsilon \ortho{\greek{ου}} and a long final eta \ortho{\greek{η}}, Attic shows \greek{κόρη} with, surprisingly enough, yet another long eta. At some point during the late Common-Attic-Ionic era, long \hellenic{/*a:/} raised to a long \hellenic{/*\ae:/}. After the dialects split, this vowel continued to raise to \hellenic{/E:/}, but it in Attic it first underwent an extra process called \german{Rückverwandlung}, or \english{Reversal}. This sound change states that any instance of \hellenic{/*\ae:/} reverts back to \hellenic{/*a:/} if it follows an \hellenic{/*r/}, \hellenic{/*e/}, or \hellenic{/*i/}. For example, where Homeric shows \hellenic{καρδίη}\ ``heart'', Attic shows \hellenic{καρδίᾱ}. The alternation between a short omicron \ortho{\greek{ο}} and a long omicron-upsilon \ortho{\greek{ου}} indicate that the root once ended in a cluster, and the quality of the long vowel indicates that the cluster was \hellenic{/*r\w/}. This consonant must have remained until after \german{Rückverwandlung} began, since its presence would block the lowering of \hellenic{/*r\ae:/} back to \hellenic{/*ra:/}. \german{Rückverwandlung} did, however, occur where there was once an intervocalic consonant \hellenic{*\w}, as in A \greek{ἐννέᾱ}\ ``nine'' $\gets$ CH \hellenic{*enne\w\greek{\shwa}}.\autocite[ἐννέα]{Beekes_2009} This confirms that \german{Rückverwandlung} occurred in two distinct phases: first after \hellenic{/*r/} and later after \hellenic{/*e/} and \hellenic{/*i/}, with \W-Elision occurring at some point between the two.

\noindent\textcolor{red}{Input chronology}
\begin{table}[htbp]
\centering

\begin{tabular}{lll|l|l}
    &                              &                                                         & \ipa{*kór\.{c}a:}   & \ipa{*né\.{c}a:}   \\
%(1) & Vowel Contractions          &\phon{eo, ea}{ō, ā}                                      &                     &                    \\
(1) &H-Elision                    &\phon{\ipa{h}}{$\textcolor{gray}{\emptyset}$}             &                     &                    \\
(2) &J-Elision                    &\phon{\ipa{j}}{$\textcolor{gray}{\emptyset}$}             &                     &                    \\
(3) &Long-Mid Vowel Lowering      &\phon{\ipa{e:, o:}}{\ipa{E:, O:}}                         &                     &                    \\
(4) &Attic Vowel Raising 1        &\phon{\ipa{a:}}{\ipa{\ae:}}                               & \ipa{*kór\.{c}\ae:} & \ipa{*né\.{c}\ae:} \\
(5) &Rückverwandlung 1            &\phonc{\ipa{\ae:}}{\ipa{a:}}{\ipa{r}\phold}               &                     &                    \\
(6) &Intervocalic \.{C}-Elision   &\phonc{\ipa{\w}}{$\textcolor{gray}{\emptyset}$}{V\phold V}&                     & \ipa{*né\ae:}      \\
(7) &Postconsonantal \.{C}-Elision&\phonc{\ipa{\w}}{$\textcolor{gray}{\emptyset}$}{C\phold}  & \ipa{*kór\ae:}      &                    \\
(8) &Rückverwandlung 2            &\phonc{\ipa{\ae:}}{\ipa{a:}}{\ipa{e(:), i(:)}\phold}      &                     & \ipa{néa:}         \\
(9) &Attic Vowel Raising 2        &\phon{\ipa{\ae:}}{\ipa{E:}}                               & \ipa{kórE:}         &                    \\
    &                             &                                                          & \textel{κόρη}       & \textel{νέᾱ}
    
\end{tabular}
\caption{Relative Chronology of Attic Sound Changes}
\label{tab:Chronology}
\end{table}
\section{Reconstructions}\label{sec:Recon}
The one-by-one nature of sound changes leaves room for automation. Regular changes mean allow for a computer to search for evidence of a process, guess at reconstructions, and test the veracity of its own guesses.  To this end, I have written three short scripts in a programming language called R, which currently search for instances of the hypothetical consonant *\w\ between vowels.

The first script is named HomeR. It is a text miner, which is to say that it reads, manipulates, and assists in the analysis of written text. Currently, it reads the entire text of the Iliad, one book at a time. HomeR then places each line into a table, and labels the lines with their Global Line Number, which is the line's position in the epic; their Book Number, which is the book from which a line is retrieved; and the Relative Line Number, which is the line's position in that book. HomeR then tokenizes the text, which is to say it splits each line into distinct tokens. A token is a textual variable, which can be defined as one of multiple levels, including lines of text, words, and individual letters. In this instance, HomeR defines a token as a word, and so places each word into its own row along with the requisite book and line numbers.

%\noindent\textcolor{red}{Input HomeR output}
\begin{table}[htbp]
\centering
\begin{tabular}{@{}rllll@{}}
\toprule
  & GlobalLine & Book & RelativeLine & Word              \\ \cmidrule(l){2-5} 
1 & 1          & 1    & 1            & \greek{μηνιν}     \\
2 & 1          & 1    & 1            & \greek{αειδε}     \\
3 & 1          & 1    & 1            & \greek{θεα}       \\
4 & 1          & 1    & 1            & \greek{πηληϊαδεω} \\
5 & 1          & 1    & 1            & \greek{αχιληος}   \\ \bottomrule
\end{tabular}
\caption{Iliad I.1 as Output by HomeR}
\label{tab:HomeR-Example}
\end{table}

The second script is named ThRax, after the Hellenistic grammarian Dionysios Thrax. It is a sound change applier, a program which holds the details of any phonological process, including the phoneme(s) to change, the constraints of the change, and the result. ThRax takes a word or series of words as an input, either from a list or the column of a table, which then pass through each programmed sound change in order, and then return as newer derivations. 

ThRax cannot assert the validity of a reconstruction on its own, since more than one possible reconstruction may give the same result. Both \hellenic{*kal\w os} and \iform[L]{kalhos}, for example, would both return \greek{κᾱλός}, a valid derivation for Homeric Greek. Prior knowledge of the different reflexes of \W-Elision in Attic and Homeric Greek is necessary to identify \iform[L]{kalhos} as an incorrect reconstruction. Some reconstructions, however, can be disqualified with a surface-level analysis of ThRax' output. The incorrect reconstruction \iform[L]{kaljos} would return \iform[L]{κᾰλλός}, which is disqualifiable at a glance.

The third and final script is named GoRgias, after the Athenian sophist and philosopher. GoRgias is a management tool, and the primary means of interfacing with HomeR and ThRax. It has four components. First, it reads the output from HomeR and sets the scope of the text to be analyzed. Second, it reads through that scope and sets the parameters to search for. Third, it transcribes the text in question into the International Phonetic Alphabet. Lastly, it guesses at a reconstruction of whatever terms match the search parameters, and sends them through ThRax. The filtered results from HomeR, the guessed reconstructions, and the results from ThRax are then bundled into a table for further analysis. Given that ThRax relies on manually defined sound change rules, the user can easily alter what point in time the scripts should reconstruct to. Any word form could be treated as older or newer depending on how far back in the chronology ThRax is programmed for. If ThRax were given every change back to the start of the Common-Hellenic period, then the input from GoRgias would essentially be treated as a Common-Hellenic form, since it would pass through all of the changes restricted to forms from that time period.  Since this paper has focused on reconstructing the hypothetical consonant *\w\ to its latest possible days, and since the elision of the consonant *\w\ predates the split between the Attic and Ionic dialects, the scripts will work to reconstruct the consonant as it was in the last days of the Common-Attic-Ionic period. 

\marginnote{Rough transition. (Minimalist?)} When searching for roots and affixes, GoRgias is instructed to reduce the scope of analysis to one scene from book VI, namely  Hector's farewell to Andromache and his prayer for Astyanax, in \bookref{Iliad}{6}{390-412; 476-481}.. It then searches through the passage and filters the results to any instances of intervocallic hiatus. The consonant *\w\ is then inserted between vowels in three phases. First, the consonant is placed between two monophthongs, such as \ipa{/e.o/} or \ipa{/a.E:/}. Second, it is placed between a diphthong and a monophthong, such as \ipa{/a\textsubarch{I}.E:/} or \ipa{/y\textsubarch{I}.o/}. The instructions to GoRgias are worded in such a way that this step will also capture hiatus between diphthongs, such as \ipa{/a\textsubarch{I}.o\textsubarch{I}/}. Lastly, the consonant is added between a monophthong and a diphthong, such as \ipa{/e.o\textsubarch{I}/}. 

%\noindent\textcolor{red}{Input GoRgias example}
\begin{table}[htbp]
\centering
\resizebox{\textwidth}{!}{%
\begin{tabular}{@{}rllllll@{}}
\toprule
  & GlobalLine & Book & RelativeLine & Word                & \W List                   & applysoundchange(\W List) \\ \cmidrule(l){2-7} 
1 & 3792       & 6    & 390          & \greek{ταμιη}       & \hellenic{tami\w ɛ:}      & \hellenic{tamiɛ:}         \\
2 & 3793       & 6    & 391          & \greek{εϋκτιμενας}  & \hellenic{e\w üktimenas}  & \hellenic{eüktimenas}     \\
3 & 3793       & 6    & 391          & \greek{αγυιας}      & \hellenic{agyĭ\w as}      & \hellenic{agyĭas}         \\
4 & 3794       & 6    & 392          & \greek{διερχομενος} & \hellenic{di\w erxomenos} & \hellenic{dierxomenos}    \\
5 & 3795       & 6    & 393          & \greek{σκαιας}      & \hellenic{skaĭ\w as}      & \hellenic{kaĭas}          \\ \bottomrule
\end{tabular}
}
\caption{Example Output of GoRgias}
\label{tab:GoRgias-Example}
\end{table}

Note that these scripts are greedy, and will generally over-capture. For instance, the denominal suffixes \greek{-ιος} from PIE \pie{*-i-os}, \greek{-ιᾰ} $\gets$ PIE \pie{*-i-h}\textit{\textsubscript{2}}, and \greek{-ίη} $\gets$ PIE \pie{-\'\i-eh}\textit{\textsubscript{2}} are all consistently captured and reconstructed as Common-Hellenic \iform[L]{-i\w os}, \iform[L]{-i\w\u{a}}, and \iform[L]{-i\w a:}. This is preferable to under-capturing, as some forms may be inappropriately excluded, such as \greek{ἰάχω} $\gets$ CH \hellenic{*(\w)i\w ak\super{h}o:} and \greek{Ἀχαιοί} $\gets$ CH  \hellenic{*ak\super{h}a\textsubarch{I}\w o\textsubarch{I}}. Analysis of GoRgias' findings indicates a list of ten roots in the passage which may have had the consonant *\w\ at the end of the Common-Attic-Ionic period.

%\noindent\textcolor{red}{Input roots list}
\begin{table}[htbp]
\centering
\begin{tabular}{@{}lllllll@{}}
\toprule
Attested &
  Reconstructed &
  PAI Root &
   &
  Attested &
  Reconstructed &
  PAI Root \\ \midrule
\greek{Σκαιάς} &
  *\hellenic{ska\textsubarch{I}\.{c}as} &
  *\lroot{\hellenic{ska\textsubarch{I}\.{c}-}} &
   &
  \greek{πάϊς} &
  *\hellenic{pa\.{c}is} &
  *\lroot{\hellenic{pa\.{c}-}} \\
\greek{θέουσα} &
  *\hellenic{t\super{h}e\.{c}o:sa} &
  *\lroot{\hellenic{t\super{h}e\.{c}-}}  &
   &
  \greek{ἰάχων} &
  *\hellenic{\.{c}i\.{c}ak\super{h}O:n} &
  *\lroot{\hellenic{\.{c}ak-}} \\
 %%%%%%%%%%%%%%%%%%%
\greek{χέουσα} &
  *\hellenic{k\super{h}e\.{c}o:sa} &
  *\lroot{\hellenic{k\super{h}e\.{c}-}} &
 %%%%%%%%%%%%%%%%%
   &
  \greek{υἱόν} &
  *\hellenic{hy\textsubarch{I}\.{c}on} &
  *\lroot{\hellenic{hy\textsubarch{I}\.{c}-}} \\
\greek{ἐλεαίρες} &
   *\hellenic{ele\.{c}a\textsubarch{I}res} &
   *\lroot{\hellenic{ele\.{c}-}} &
   &
  \greek{ἐρύετο} &
  *\hellenic{\.{c}eryeto} &
  *\lroot{\hellenic{\.{c}eru-}} \\
\greek{Ἀχαιοί} &
  *\hellenic{ak\super{h}a\textsubarch{I}\.{c}o\textsubarch{I}} &
  *\lroot{\hellenic{ak\super{h}a\textsubarch{I}\.{c}-}} &
   &
   &%\greek{θεοὶ} &
   &%*\hellenic{t\super{h}e\.{c}o\textsubarch{I}} &
   \\ \bottomrule
\end{tabular}
\caption[Reconstructed Common-Attic-Ionic Roots]{Reconstructed Common-Attic-Ionic Roots}
\label{tab:recon-roots}
\end{table}

\subsection{Suffixes}

%\noindent\textcolor{red}{Love yourself and write a para for -wos. Here, I'll even get you started!}
%for example, -ος in κεραός "horned" can be differentiated from the typical nominative -ος in δρόμος "race". 

Together, the scripts also identify a number of suffixes with the consonant *\w.
This includes some more common suffixes, such as \greek{-ος, -αρ, -ων}, and,
in the passage specifically, \greek{-εις}.

The adjectival suffix \greek{-ος} is distinguishable from the nominal \greek{-ος} in two key ways. First, in vowel-stem terms, the suffix-initial omicron \ortho{\greek{ο}} conspicuously does not contract with the stem-final vowel, such as in \greek{κεραός} ``horned''. Second, in consonant-stem terms, Homeric shows long vowels corresponding to Attic short vowels, such as in A \greek{ὅλος} and H \greek{οὖλος}, or A \greek{ξένος} and H \greek{ξεῖνος}. There is no lowering of mid-high long vowels, implying compensatory lengthening from the loss of *\w, and giving the Common-Attic-Ionic suffix \hellenic{*-\w os}

The suffixes \greek{-αρ} and \greek{-ων}, found in \greek{στέαρ} and \greek{πίων},
are both confirmed by examining other derivations from shared roots. \greek{Στέαρ}
shares a root with both \greek{στάσις} and \greek{ἵστημι}, which provide
the root \groot{στ\ea-}, with the alternations in vowel quality being a result
of IE ablaut; \footnote{Cf. Smyth 35-56} while \greek{πίων} shares its root with
\greek{πιμέλη}. Given that the forms \greek{στέαρ} and \greek{πίων} show signs
of a missing consonant, while their cognate derivatives do not, it can be
confidently argued that the consonant was a part of the suffix, not the roots.
Both forms disqualify an intervocalic \hellenic{*-h-} or \hellenic{*-j-}, given the lack of any
discernible effect on the preceding vowel following elision.
No compensatory lengthening occurred, disqualifying H-Elision, and no
diphthongs were formed, disqualifying J-Elision. The remaining options, then,
are the forms \greek{στέαρ} $\gets$ CAI \hellenic{*st$\nicefrac{e}{a}$-\w ar} and
\hellenic{*pi-\w O:n}. 

The base form of \hellenic{*-\w O:n}, however, must have had a short vowel.
The suffix shows quantitative alternation between cases, with the
Genitive Singular \greek{-ονος} and the Dative Singular \greek{-ονι},
suggesting that the omicron \ortho{\greek{ο}} is lengthened when in the
word-final position in much the same way as the final vowel in
\greek{πάτηρ} and \greek{ἄνηρ}. This gives the suffix' base form
\greek{-on-}, reconstructed as PAI \hellenic{*-\w\u{o}n-}.

%\textcolor{red}{\textel{-ων} endings here}
\begin{table}[htbp]
	\centering
        \begin{tabular}{@{}lccc@{}}
        \toprule
             & \multicolumn{2}{c}{M \& F}       & N           \\ \cmidrule(l){2-4} 
        Nom. & \multicolumn{2}{c}{\textel{ων}}  & \textel{ον} \\
        Gen. & \multicolumn{3}{c}{ονος}                       \\
        Dat. & \multicolumn{3}{c}{ονι}                        \\
        Acc. & \multicolumn{2}{c}{\textel{ονα}} & \textel{ον} \\
        Voc. & \multicolumn{3}{c}{ον}                         \\ \bottomrule
        \end{tabular}
	\caption{\textel{-ων} Singular Case Endings}
	\label{tab:ων-Endings}
\end{table}

The suffix \greek{-εις} can be reconstructed with an initial *\w.
It is attested twice times in the passage, in the words \greek{ὑληέσσῃ} and \greek{βροτόεντα}.
It can be reasonably expected that the suffix begin with a consonant
given the construction of the word \greek{βροτ-ό-εντα}, attested with
an omicron \ortho{\greek{ο}} inserted as a connecting vowel, which only
occurred when the addition of an affix to a stem would have created a
consonant cluster. The omicron, then, must have been added to separate
the root-final tau \ortho{\greek{τ}} and whatever consonant followed.
If that consonant had been \hellenic{*-h-} or \hellenic{*-j-}, then its elision would have
allowed for contraction between the connecting omicron \ortho{\greek{ο}}
and the suffix-initial epsilon \ortho{\greek{ε}} giving the form
\iform[G]{βροῦντα}. The initial consonant, then, must have again been the
hypothetical consonant \w.

Much like the suffix \greek{-ων}, the Nominative forms of \greek{-εις}
obfuscate the base form. Looking at the Genitive, Dative, and Accusative
forms -- \greek{-εντος}, \greek{-εντι}, and \greek{-εντα}, respectively --
show the base form \greek{-εντ-}. The Nominative form can be explained
as the result of two successive consonant contractions, giving the
process \greek{-εντς} $\to$ \greek{-ενς} $\to$ \greek{-εις}, with
the long epsilon \ortho{\greek{ει}}
arising through compensatory
lengthening. This indicates the CAI form \hellenic{*-\w ents}.

\marginnote{Can you tell this was just slapped in here?}
Lastly, and in a similar manner to \hellenic{*-\w os}, hiatus also indicates the consonant *\w\ in the suffix \greek{-ως}, as in \greek{Τρώς} ``A Trojan''. However, unlike the other suffixes found in the passage, he Genitive and Dative singular forms \greek{Τρωός} and \greek{Τρῶϊ}, indicate a suffix-final consonant. In the case of \greek{Τρώς}, other forms were built with secondary suffixes (suffixes appended onto a prior suffix), such as in \greek{Τρωϊκός} ``Trojan'', \greek{Τρώϊος} ``Trojan man'', and \greek{Τρώϊα} ``Trojan women''. This strategy is also responsible for the form \greek{ἡρωΐνη} $\gets$ CAI \hellenic{hE:rO:\w\'\i n\ae}

\begin{table}[htbp]
\centering
\begin{tabular}{@{}llll@{}}
\toprule
Attested     & Reconstructed       &                          &                                     \\ \midrule
             & M                   & F                        & N                                   \\ \cmidrule(l){2-4} 
\greek{-ος } & \hellenic{*-\w o-s} & \hellenic{*-\w e-}       & \hellenic{*-\w o-n}                 \\
\greek{-αρ}  &                     &                          & \hellenic{*-\w a$\nicefrac{r}{t}$-} \\
\greek{-ων}  & \multicolumn{2}{l}{\hellenic{*-\w\u={o}n-}}    & \hellenic{*-\w\u{o}n-}              \\
\greek{-εις} & \hellenic{*-\w \u{e}nt-} & \hellenic{*-\w et't'-\u={a}} & \hellenic{*-\w $\nicefrac{en}{ent-}$} \\
\greek{-ως}  & \hellenic{*\=o\w-s} & \hellenic{*\=o\w-in-\=a} &                                     \\ \bottomrule
\end{tabular}
\caption{Reconstructed Common-Attic-Ionic Suffixes}
\label{tab:recon-suffixes}
\end{table}

%\dpara{(e)u}
GoRgias also demonstrates a connection between the two
u-type agent suffixes, \greek{-υς} and \greek{-ευς}. The script, when
instructed to examine the chosen passage, captures \greek{υἱόν}, the
Accusative Singular of \greek{υἱύς}; when instructed to examine the
Iliad as a whole, it captures such terms as \greek{ἡδύς},
\greek{Ἀχιλλεύς}, and \greek{βασιλεύς}. The declension of the u-type
endings show a strong concordance in their endings (though
their exact form varies somewhat to satisfy metrical requirements).
This indicates an etymological relationship between the two, where
\greek{-ευς}, with an epsilon \ortho{\greek{ε}}, is the
thematic form of \greek{-υς}, which lacks any such vowel. 

%\textcolor{red}{\greek{-(ε)υς} endings here}
\begin{table}[htbp]
	\centering
	\begin{tabular}{@{}lcc@{}}
		\toprule
		     & \greek{-υ-} & \greek{-ευ-}       \\ \cmidrule(l){2-3}
		Nom. & \textel{υς} & \textel{ευς}       \\
		Gen. & \multicolumn{2}{c}{\textel{εος}} \\
		Dat. & \multicolumn{2}{c}{\textel{εϊ}}  \\
		Acc. & \textel{υν} & \textel{εα}        \\
		Voc. & \textel{υ}  & \textel{ευ}        \\
		\bottomrule
	\end{tabular}
	\caption{\textel{-υ}- and \textel{-ευ}-Type Singular Endings}
	\label{tab:U-Endings}
\end{table}

%\dpara{BRING IT HOME}
These endings are also noteworthy for what they indicate about the
consonant *\w: the position of the intervocalic hiatus in these
endings can indicate the specific quality, the actual pronunciation,
of the hypothetical consonant.
Take the Dative Singular ending, variously attested as \greek{-ηϊ}
or \greek{-εϊ}. The hiatus stands exactly where the Nominative
singular shows the upsilon \ortho{\greek{υ}}, which apparently
disappeared in declension -- the dative, otherwise, would read as
\iform[G]{-ηυϊ} or \iform[G]{-ευι}. It has already been determined
that the consonant was likely either an approximant or a fricative,
and given that it was related to the upsilon, and that some vowels
have a relative approximant, specifically in the form of a relative
semivowel, the pronunciation of the upsilon
can potentially identify the consonant \w's specific pronunciation.

However, given the historical nature of this relationship, it is not
useful to examine the pronunciation of upsilon \ortho{υ} as it was
pronounced in the classical era. At this point, both Attic and
Ionic speakers pronounced this as the front round vowel \ipa{/y/},\footnote{Cf. Fr \french{u} and De \german{\"u}} however it
was fronted some time between 700 and 400 \textsc{b.c.e.}, moving
from its original pronunciation as a back round vowel \ipa{/u/}.\autocite[529]{malikouti-drachman_bortone_2015}
In this instance, then, the vowel upsilon \ortho{υ} needs to be
analyzed as that back round vowel. 

%\textcolor{red}{[Semivowel relationship chart]}
\begin{table}[htbp]
\centering
\caption{Greek Phonemes with Relative Semivowels}
\label{tab:semivowels}
\begin{tabular}{lll}
Greek                       & Vowel   & Semivowel \\
\textel{ι}                  & \ipa{i} & \ipa{j}   \\
\multirow{2}{*}{\textel{υ}} & \ipa{y} & \ipa{4}   \\
                            & \ipa{u} &          
\end{tabular}
\end{table}

With this in mind, the behavior of the upsilon \ortho{υ} indicates
that the historic consonant likely was pronounced as \ipa{/w/},
the sound of the English letter double-u \ortho{\english{w}}. 
The pronunciation of the two sounds are near-identical, excepting
for differences in quantity. That is, the vowel \ipa{/u/} is typically
held longer than the semivowel \ipa{/w/}. Even the specific features
of these sounds have a near perfect correlation,
except that \ipa{/w/} is labeled as a consonant and \ipa{/u/} a vowel.
This would explain the behavior of the upsilon \ortho{\greek{υ}} in u-type
endings, with the short, intervocalic \ipa{/u/} being re-bracketed, being
understood not as the end of the prior syllable but the beginning of the
latter, and then re-analyzed as a semivowel, which was subsequently elided.
The dative, then, progressed from the Common-Attic-Ionic
\hellenic{*-$\nicefrac{e}{\bar{e}}$\textsubarch{u}.i} $\to$
\hellenic{*-$\nicefrac{e}{\bar{e}}$.\textsubarch{u}i} $=$
\hellenic{*-$\nicefrac{e}{\bar{e}}$.wi} $\to$ Homeric
-\greek{$\nicefrac{\textrm{\greek{ε}}}{\textrm{\greek{η}}}$ϊ}

\begin{figure}
    \centering
        \ipa{w} \\
        \phonfeat{
        $+$ Sonorant \\
        $+$ Back \\
        $+$ High \\
        $+$ Labial \\
        $+$ Continuant \\
        $+$ Voiced
    }
    \caption{Featurs of \ipa{w}}
    \label{fig:FeaturesIII}
\end{figure}
\clearpage
\printbibheading
\printbibliography[heading=subbibliography,title={Texts Cited}]
\clearpage
\begin{appendixes}
    \section[Hector's Farewell]{Hector's Farewell in Reconstructed Common-Attic-Ionic}
        % Book 6
\begin{Versi}[390]
\hellenic{\^E: \r{r}a gun\`\ae: tam\'\i\ae:, h\`o d' ap\'et't'uto d\'O:matos h\'ektO:r}\\
%ἦ ῥα γυνὴ ταμίη, ὃ δ' ἀπέσσυτο δώματος Ἕκτωρ\\
\hellenic{t\`\ae n aut\`\ae n hod\`on \'a\textsubarch{u}tis e\"uktim\'ena:s kat' agu\textsubarch{I}\'a:s}\\
%τὴν αὐτὴν ὁδὸν αὖτις ἐϋκτιμένας κατ' ἀγυιάς.\\
\hellenic{\'e\textsubarch{u}te p\'ula:s \'\i:ka:ne dierk\super{h}\'omenos m\'ega w\'astu}\\
%εὖτε πύλας ἵκανε διερχόμενος μέγα άστυ\\
\hellenic{skaiw\'as, t\'\ae:\textsubarch{I} ar' \'emelle dieks\'\i mena\textsubarch{I} ped\'\i onde}\\
%Σκαιάς, τῇ ἄρ' ἔμελλε διεξίμεναι πεδίονδε\\
\hellenic{\'ent\super{h}' \'alok\super{h}os pol\'udO:ros enant\'\i\ae: \greek{ἦλθε} t\super{h}\'ewo:sa}\\
%ἔνθ' ἄλοχος πολύδωρος ἐναντίη ἦλθε θέουσα\\ % Possibly
\hellenic{androm\'ak\super{h}\ae: t\super{h}ug\'atE:r megal\'E:toros E:et\'\i O:nos}\\
%Ἀνδρομάχη θυγάτηρ μεγαλήτορος Ἠετίωνος\\ % Possibly suffix -won, cf pion
\hellenic{E:et\'\i O:n h\`os \'ena\textsubarch{I}en hup\`o pl\'akO:\textsubarch{I} hu:l\ae:w\'et't'\ae:\textsubarch{I}}\\
%Ἠετίων ὃς ἔναιεν ὑπὸ Πλάκῳ ὑληέσσῃ\\ % Suffix -eis (-wents), cf haimatoweis
\hellenic{t\super{h}\'E:b\ae:\textsubarch{I} hupoplak\'\i\ae\textsubarch{I} \greek{κιλίκεσσ' ἄνδρεσσιν} wan\'at't'O:n}\textit{:}\\
%Θήβῃ Ὑποπλακίῃ Κιλίκεσσ' ἄνδρεσσιν ἀνάσσων:\\
\hellenic{t\^o: per d\`E: t\super{h}ug\'atE:r \'ek\super{h}et\super{h}' h\'ektori k\super{h}alkokorust\'\ae:\textsubarch{I}.}\\
%τοῦ περ δὴ θυγάτηρ ἔχεθ' Ἕκτορι χαλκοκορυστῇ.\\
\hellenic{h\'\ae: ho\textsubarch{I} \'epe:t' \greek{ἤντησ'}, h\'ama d' amp\super{h}\'\i polos k\'\i en aut\'\ae:\textsubarch{I}}\\
%ἥ οἱ ἔπειτ' ἤντησ', ἅμα δ' ἀμφίπολος κίεν αὐτῇ\\ % ἀμφίπολος < ἀμφίκϝολος -- kw -> p/t too early for PAI
\hellenic{p\'a:wida k\'olpO:\textsubarch{I} \'ek\super{h}o:s' atal\'ap\super{h}rona \greek{νήπιον} \'a\textsubarch{u}tO:s}\\
%παῖδ' ἐπὶ κόλπῳ ἔχουσ' ἀταλάφρονα νήπιον αὔτως\\ % -φρων< φρήν <- *gʷʰren-; νήπιος <- ν(ᾱ/η)ϝέπ-ι-ος
Ἑκτορίδην ἀγαπητὸν ἀλίγκιον ἀστέρι καλῷ,\\ % Haven't done this row yet
τόν ῥ' Ἕκτωρ καλέεσκε Σκαμάνδριον, αὐτὰρ οἱ ἄλλοι\\
Ἀστυάνακτ': οἶος γὰρ ἐρύετο Ἴλιον Ἕκτωρ.\\
ἤτοι ὃ μὲν μείδησεν ἰδὼν ἐς παῖδα σιωπῇ:\\
Ἀνδρομάχη δέ οἱ ἄγχι παρίστατο δάκρυ χέουσα,\\
ἔν τ' ἄρα οἱ φῦ χειρὶ ἔπος τ' ἔφατ' ἔκ τ' ὀνόμαζε:\\
δαιμόνιε φθίσει σε τὸ σὸν μένος, οὐδ' ἐλεαίρεις\\
παῖδά τε νηπίαχον καὶ ἔμ' ἄμμορον, ἣ τάχα χήρη\\
σεῦ ἔσομαι: τάχα γάρ σε κατακτανέουσιν Ἀχαιοὶ\\
πάντες ἐφορμηθέντες: ἐμοὶ δέ κε κέρδιον εἴη\\
σεῦ ἀφαμαρτούσῃ χθόνα δύμεναι: οὐ γὰρ ἔτ' ἄλλη\\
ἔσται θαλπωρὴ ἐπεὶ ἂν σύ γε πότμον ἐπίσπῃς\\ \ladd{\ldots}\\
\end{Versi}
\begin{Versi}[466]
%ὣς εἰπὼν
\ladd{\ldots} οὗ παιδὸς ὀρέξατο φαίδιμος Ἕκτωρ:\\
ἂψ δ' ὃ πάϊς πρὸς κόλπον ἐϋζώνοιο τιθήνης\\
ἐκλίνθη ἰάχων πατρὸς φίλου ὄψιν ἀτυχθεὶς\\
ταρβήσας χαλκόν τε ἰδὲ λόφον ἱππιοχαίτην,\\
δεινὸν ἀπ' ἀκροτάτης κόρυθος νεύοντα νοήσας.\\
ἐκ δ' ἐγέλασσε πατήρ τε φίλος καὶ πότνια μήτηρ:\\
αὐτίκ' ἀπὸ κρατὸς κόρυθ' εἵλετο φαίδιμος Ἕκτωρ,\\
καὶ τὴν μὲν κατέθηκεν ἐπὶ χθονὶ παμφανόωσαν:\\
αὐτὰρ ὅ γ' ὃν φίλον υἱὸν ἐπεὶ κύσε πῆλέ τε χερσὶν\\
εἶπεν ἐπευξάμενος Διΐ τ' ἄλλοισίν τε θεοῖσι:\\
Ζεῦ ἄλλοι τε θεοὶ δότε δὴ καὶ τόνδε γενέσθαι\\
παῖδ' ἐμὸν ὡς καὶ ἐγώ περ ἀριπρεπέα Τρώεσσιν,\\
ὧδε βίην τ' ἀγαθόν, καὶ Ἰλίου ἶφι ἀνάσσειν:\\
καί ποτέ τις εἴπῃσι πατρός δ' ὅ γε πολλὸν ἀμείνων\\
ἐκ πολέμου ἀνιόντα: φέροι δ' ἔναρα βροτόεντα\\
κτείνας δήϊον ἄνδρα, χαρείη δὲ φρένα μήτηρ.\\
\end{Versi}
\end{appendixes}

%\vspace*{1\baselineskip}
%\textcolor{red}{\hrule}

Together, the scripts also identify a number of suffixes with the consonant *\w.
This includes some more common suffixes, such as \textel{-αρ, -ων}, and,
in the passage specifically, \textel{-εις}.
Some, such as \textel{-ος}, can be confidently distinguished from similar forms --
here, the o-stem nominative \textel{-ος} -- by their extraneous hiatus,
but others require more investigation.

The suffixes \textel{-αρ} and \textel{-ων}, found in \textel{στέαρ} and \textel{πίων},
are both confirmed by examining other derivations from shared roots. \textel{Στέαρ}
shares a root with both \textel{στάσις} and \textel{ἵστημι}, which provide
the root \groot{στ\ea-}, with the alternations in vowel quality being a result
of IE ablaut; \footnote{Cf. Smyth 35-56} while \textel{Πίων} shares its root with
\textel{πιμέλη}. Given that the forms \textel{στέαρ} and \textel{πίων} show signs
of a missing consonant, while their cognate derivatives do not, it can be
confidently argued that the consonant was a part of the suffix, not the roots.
Both forms disqualify an intervocalic *-s- or *-j-, given the lack of any
discernible effect on the preceding vowel following elision.
No compensatory lengthening occurred, disqualifying s-elision, and no
diphthongs were formed, disqualifying j-elision. The remaining options, then,
are the forms \textel{στέαρ} $\gets$ CAI *st$\nicefrac{e}{a}$-\w ar and
*pi-\w \=on. 

The base form of *-\w\=on, however, must have had a short vowel.
The suffix shows quantitative alternation between cases, with the
Genitive Singular \textel{-ονος} and the Dative Singular \textel{-ονι},
suggesting that the omicron \spell{\textel{ο}} is lengthened when in the
word-final position in much the same way as the final vowel in
\textel{πάτηρ} and \textel{ἄνηρ}. This gives the suffix' base form
\textel{-on-}, reconstructed as PAI *-\w\u{o}n-.

\textcolor{red}{\textel{-ων} endings here}

The suffix \textel{-εις} can be reconstructed with an initial *\w.
It is attested three times in the passage, in the words \textel{ὑληέσσῃ},
\textel{Τρώεσσιν}, and \textel{βροτόεντα}.
It can be reasonably expected that the suffix begin with a consonant
given the construction of the word \textel{βροτ-ό-εντα}, attested with
an omicron \spell{\textel{ο}} inserted as a connecting vowel, which only
occurred when the addition of an affix to a stem would have created a
consonant cluster. The omicron, then, must have been added to separate
the root-final tau \spell{\textel{τ}} and whatever consonant followed.
If that consonant had been *-s- or *-j-, then its elision would have
allowed for contraction between the connecting omicron \spell{\textel{ο}}{o}
and the suffix-initial epsilon \spell{\textel{ε}} giving the form
\iform[G]{βροῦντα}. The initial consonant, then, must have again been the
hypothetical consonant \w.

Much like the suffix \textel{-ων}, the Nominative forms of \textel{-εις}
obfuscate the base form. Looking at the Genitive, Dative, and Accusative
forms -- \textel{-εντος}, \textel{-εντι}, and \textel{-εντα}, respectively --
show the base form \textel{-εντ-}. The Nominative form can be explained
as the result of two successive consonant contractions, giving the
process \textel{-εντς} $\to$ \textel{-εινς} $\to$ \textel{-εις}, with
the long epsilon \spell{\textel{ει}}{e:}
arising through compensatory
lengthening. This indicates the CAI form *-\w ents.

\dpara{(e)u} GoRgias also demonstrates a connection between the two
u-type agent suffixes, \textel{-υς} and \textel{-ευς}. The script, when
instructed to examine the chosen passage, captures \textel{υἱόν}, the
Accusative Singular of \textel{υἱύς}; when instructed to examine the
Iliad as a whole, it captures such terms as \textel{ἡδύς},
\textel{Ἀχιλλεύς}, and \textel{βασιλεύς}. The declension of the -type
endings show a strong concordance in their endings (though
their exact form varies somewhat to satisfy metrical requirements).
This indicates an etymological relationship between the two, where
\textel{-ευς}, with a present epsilon \spell{\textel{ε}}{e}, is the
thematic form of \textel{-υς}, which lacks any such vowel. 

\textcolor{red}{\textel{-(ε)υς} endings here}

\dpara{BRING IT HOME}
These endings are also noteworthy for what they indicate about the
consonant *\w: the position of the intervocalic hiatus in these
endings can indicate the specific quality, the actual pronunciation,
of the hypothetical consonant.
Take the Dative Singular ending, variously attested as \textel{-ηϊ}
or \textel{-εϊ}. The hiatus stands exactly where the Nominative
singular shows the upsilon \spell{\textel{υ}}{y}, which apparently
disappeared in declension -- the dative, otherwise, would read as
\iform[G]{-ηυϊ} or \iform[G]{-ευι}. It has already been determined
that the consonant was likely either an approximant or a fricative,
and given that it was related to the upsilon, and that some vowels
have a relative approximant, specifically in the form of a relative
semivowel, the pronunciation of the upsilon
can potentially identify the consonant \w's specific pronunciation.

However, given the historical nature of this relationship, it is not
useful to examine the pronunciation of upsilon \ortho{υ} as it was
pronounced in the classical era. At this point, both Attic and
Ionic speakers pronounced this as the front round vowel \ipa{/y/}
(cf. Fr \ortho{\textrm{u}} and De \ortho{\textrm{\"u}}, however it
was fronted some time between 700 and 400 \textsc{b.c.e.}, moving
from its original pronunciation as a back round vowel \ipa{/u/} (cf.
Fr \ortho{\textrm{ou}} and De \ortho{\textrm{u}}).\autocite[529]{malikouti-drachman_bortone_2015}
In this instance, then, the vowel upsilon \ortho{υ} needs to be
analyzed as that back round vowel. 

\textcolor{red}{[Semivowel relationship chart]}
%\begin{table}[htbp]
\centering
\caption{Greek Phonemes with Relative Semivowels}
\label{tab:semivowels}
\begin{tabular}{lll}
Greek                       & Vowel   & Semivowel \\
\textel{ι}                  & \ipa{i} & \ipa{j}   \\
\multirow{2}{*}{\textel{υ}} & \ipa{y} & \ipa{4}   \\
                            & \ipa{u} &          
\end{tabular}
\end{table}

With this in mind, the behavior of the upsilon \ortho{υ} indicates
that the historic consonant likely was pronounced as \ipa{/w/},
the sound of the Gnglish letter double-u \ortho{\textrm{w}}. 
The pronunciation of the two sounds are near-identical, excepting
for differences in quantity -- that is, the vowel \ipa{/u/} is typically
held longer than the semivowel \ipa{/w/}. In fact, the specific features
of these sounds -- literally the physical methods used in their pronunciation,
such has tongue position and tension -- have a near perfect correlation,
except that \ipa{/w/} is labeled as a consonant and \ipa{/u/} a vowel.
This would explain the behavior of the upsilon \ortho{υ} in u-type
endings, with the short, intervocalic /u/ being re-bracketed, being
understood not as the end of the prior syllable but the beginning of the
latter, and then re-analyzed as a semivowel, which was subsequently elided.
The dative, then, progressed from the Common-Attic-Ionic
*-$\nicefrac{e}{\bar{e}}$u.i $\to$ *-$\nicefrac{e}{\bar{e}}$.ui $\to$
*-$\nicefrac{e}{\bar{e}}$.wi $\to$ Homeric
-\textel{$\nicefrac{\textrm{\textel{ε}}}{\textrm{\textel{η}}}$ϊ}

\textcolor{red}{[Input final reconstruction!!!?!?!1`!]}



\end{document}