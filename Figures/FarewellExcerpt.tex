% Book 6
\begin{Versi}[390]
\hellenic{\^E: \r{r}a gun\`\ae: tam\'\i\ae:, \inipartvoiceless{\`o} d' ap\'e\v{c}\v{c}uto d\'O:matos \inipartvoiceless{\'e}ktO:r}\\
%ἦ ῥα γυνὴ ταμίη, ὃ δ' ἀπέσσυτο δώματος Ἕκτωρ\\
\hellenic{t\`\ae n a\textsubarch{u}t\`\ae n \inipartvoiceless{o}d\`on \'a\textsubarch{u}tis e\"uktim\'ena:s kat' agu\textsubarch{I}\'a:s}\\
%τὴν αὐτὴν ὁδὸν αὖτις ἐϋκτιμένας κατ' ἀγυιάς.\\
\hellenic{\'e\textsubarch{u}te p\'ula:s \'\i:ka:ne dierk\super{h}\'omenos m\'ega w\'astu}\\
%εὖτε πύλας ἵκανε διερχόμενος μέγα άστυ\\
\hellenic{ska\textsubarch{I}w\'as, t\'\ae:\textsubarch{I} ar' \'emelle dieks\'\i mena\textsubarch{I} ped\'\i onde}\\
%Σκαιάς, τῇ ἄρ' ἔμελλε διεξίμεναι πεδίονδε\\
\hellenic{\'ent\super{h}' \'alok\super{h}os pol\'udO:ros enant\'\i\ae: \^E:t\super{h}e t\super{h}\'ewo:sa}\\
%ἔνθ' ἄλοχος πολύδωρος ἐναντίη ἦλθε θέουσα\\ % Possibly
\hellenic{androm\'ak\super{h}\ae: t\super{h}ug\'atE:r megal\'E:toros E:et\'\i O:nos}\\
%Ἀνδρομάχη θυγάτηρ μεγαλήτορος Ἠετίωνος\\ % Possibly suffix -won, cf pion
\hellenic{E:et\'\i O:n \inipartvoiceless{\`o}s \'ena\textsubarch{I}en \inipartvoiceless{u}p\`o pl\'akO:\textsubarch{I} \inipartvoiceless{u}:l\ae:w\'e\v{c}\v{c}\ae:\textsubarch{I}}\\
%Ἠετίων ὃς ἔναιεν ὑπὸ Πλάκῳ ὑληέσσῃ\\ % Suffix -eis (-wents), cf haimatoweis
\hellenic{t\super{h}\'E:b\ae:\textsubarch{I} \inipartvoiceless{u}poplak\'\i\ae\textsubarch{I} kil\'\i kets' \'andretsi wan\'a\v{c}\v{c}O:n}\textit{:}\\
%Θήβῃ Ὑποπλακίῃ Κιλίκεσσ' ἄνδρεσσιν ἀνάσσων:\\
\hellenic{t\^o: per d\`E: t\super{h}ug\'atE:r \'ek\super{h}et\super{h}' \inipartvoiceless{\'e}ktori k\super{h}alkokorust\'\ae:\textsubarch{I}.}\\
%τοῦ περ δὴ θυγάτηρ ἔχεθ' Ἕκτορι χαλκοκορυστῇ.\\
\hellenic{\inipartvoiceless{\'\ae}: \inipartvoiceless{o}\textsubarch{I} \'epe:t' \'ant\ae:s', \inipartvoiceless{\'a}ma d' amp\super{h}\'\i polos k\'\i en aut\'\ae:\textsubarch{I}}\\
%ἥ οἱ ἔπειτ' ἤντησ', ἅμα δ' ἀμφίπολος κίεν αὐτῇ\\ % ἀμφίπολος < ἀμφίκϝολος -- kw -> p/t too early for PAI
\hellenic{p\'a:wida k\'olpO:\textsubarch{I} \'ek\super{h}o:s' atal\'ap\super{h}rona n\'\ae:pion \'a\textsubarch{u}tO:s}\\
%παῖδ' ἐπὶ κόλπῳ ἔχουσ' ἀταλάφρονα νήπιον αὔτως\\ % -φρων< φρήν <- *gʷʰren-; νήπιος <- ν(ᾱ/η)ϝέπ-ι-ος
\hellenic{\inipartvoiceless{e}ktor\'\i d\ae:s agapE:t\`on al\'\i Nkion ast\'eri kal\'O:\textsubarch{I}}\\
%Ἑκτορίδην ἀγαπητὸν ἀλίγκιον ἀστέρι καλῷ,\\ % Haven't done this row yet
\hellenic{t\'on \r{r}' \inipartvoiceless{\'e}ktO:r kal\'eeske skam\'andrion, a\textsubarch{u}t\`ar \inipartvoiceless{o}\textsubarch{I} \'allo\textsubarch{I}}\\
%τόν ῥ' Ἕκτωρ καλέεσκε Σκαμάνδριον, αὐτὰρ οἱ ἄλλοι\\
\hellenic{wastuw\'anakt'}\textit{:} \hellenic{\'o\textsubarch{I}wos g\`ar wer\'ueto w\'\i lion \inipartvoiceless{\'e}ktO:r}\\
%Ἀστυάνακτ': οἶος γὰρ ἐρύετο Ἴλιον Ἕκτωρ.\\
%ἤτοι ὃ μὲν μείδησεν ἰδὼν ἐς παῖδα σιωπῇ:\\
%Ἀνδρομάχη δέ οἱ ἄγχι παρίστατο δάκρυ χέουσα,\\
%ἔν τ' ἄρα οἱ φῦ χειρὶ ἔπος τ' ἔφατ' ἔκ τ' ὀνόμαζε:\\
%δαιμόνιε φθίσει σε τὸ σὸν μένος, οὐδ' ἐλεαίρεις\\
%παῖδά τε νηπίαχον καὶ ἔμ' ἄμμορον, ἣ τάχα χήρη\\
%σεῦ ἔσομαι: τάχα γάρ σε κατακτανέουσιν Ἀχαιοὶ\\
%πάντες ἐφορμηθέντες: ἐμοὶ δέ κε κέρδιον εἴη\\
%σεῦ ἀφαμαρτούσῃ χθόνα δύμεναι: οὐ γὰρ ἔτ' ἄλλη\\
%ἔσται θαλπωρὴ ἐπεὶ ἂν σύ γε πότμον ἐπίσπῃς\\ \ladd{\ldots}\\
\end{Versi}
%\begin{Versi}[466]
%ὣς εἰπὼν
%\ladd{\ldots} οὗ παιδὸς ὀρέξατο φαίδιμος Ἕκτωρ:\\
%ἂψ δ' ὃ πάϊς πρὸς κόλπον ἐϋζώνοιο τιθήνης\\
%ἐκλίνθη ἰάχων πατρὸς φίλου ὄψιν ἀτυχθεὶς\\
%ταρβήσας χαλκόν τε ἰδὲ λόφον ἱππιοχαίτην,\\
%δεινὸν ἀπ' ἀκροτάτης κόρυθος νεύοντα νοήσας.\\
%ἐκ δ' ἐγέλασσε πατήρ τε φίλος καὶ πότνια μήτηρ:\\
%αὐτίκ' ἀπὸ κρατὸς κόρυθ' εἵλετο φαίδιμος Ἕκτωρ,\\
%καὶ τὴν μὲν κατέθηκεν ἐπὶ χθονὶ παμφανόωσαν:\\
%αὐτὰρ ὅ γ' ὃν φίλον υἱὸν ἐπεὶ κύσε πῆλέ τε χερσὶν\\
%εἶπεν ἐπευξάμενος Διΐ τ' ἄλλοισίν τε θεοῖσι:\\
%Ζεῦ ἄλλοι τε θεοὶ δότε δὴ καὶ τόνδε γενέσθαι\\
%αῖδ' ἐμὸν ὡς καὶ ἐγώ περ ἀριπρεπέα Τρώεσσιν,\\
%ὧδε βίην τ' ἀγαθόν, καὶ Ἰλίου ἶφι ἀνάσσειν:\\
%καί ποτέ τις εἴπῃσι πατρός δ' ὅ γε πολλὸν ἀμείνων\\
%ἐκ πολέμου ἀνιόντα: φέροι δ' ἔναρα βροτόεντα\\
%κτείνας δήϊον ἄνδρα, χαρείη δὲ φρένα μήτηρ.\\
%\end{Versi}

\subsection{Notes on the Reconstruction}

\paragraph{VI.390: \greek{ἀπέσσυτο}} Per Sihler, ``It appears that \textipa{/\v{c}\v{c}/} remained as the last of the distinctly palatalized consonants.''\autocite[198.a]{Sihler_1995} Given the differing reflexes in Attic and Ionic, \ortho{\greek{ττ}} and \ortho{\greek{σσ}} respectively, I think it likely that this consonant survived at least until the two dialects split, and so I have left it in the reconstruction.

\paragraph{VI.390: \greek{Ἕκτωρ}} In deciding to use the International Phonetic Alphabet for transcribing my reconstruction, I gave myself the opportunity to assert my views on the \textit{spiritus asper}, or \english{rough breathing}, found in many word-initial vowels. In short, I firmly believe that the \textit{spiritus asper} should be analyzed as an initial,\footnote{``Initial'' here meaning at the beginning of a phoneme not of a phrase.} partial devoicing of a vowel. I think it clear that an Attic or Ionic speaker would agree, given that a word-initial \textit{spiritus asper} does not block apocope as a true consonant would. For instance, if it were a representation of the consonant \ipa{/h/}, then that consonant would block the apocope in \greek{ἤντησ᾽, ἅμα}. In keeping with this attitude, I have represented \textit{spiritus asper} with an open parenthesis and a ring, as seen in PAI \hellenic{*\inipartvoiceless{o}} for A/H \greek{ὁ}, which represents that partial initial devoicing. 

\paragraph{VI.397: \greek{Κιλίκεσσ᾽}} Also per Sihler, the \AE olic-style Dative plural \ortho{\greek{-σσι}} points to the later Common-Hellenic cluster \hellenic{*-ts-}, from the older \hellenic{*-ty-} or \hellenic{*-t\super{h}y-}.\autocite[196]{Sihler_1995} 

\paragraph{VI.399: \greek{ἤντησ᾽}} The form \greek{ἤντησ᾽} provides something of a problem. First, and easiest, it is the A\"orist of \greek{ἀντάω}, built as \greek{ε-αντησ-ε}. Fortunately, the basic story of the Iliad either predates the development of the epsilon augment or the author(s) found themself unwilling to care about it, and so it is frequently dropped when metrically convenient. This handily removes the first eta, which is metrically irrelevant, since the remaining vowel is still long by position. As for the second eta, that arises from a long \hellenic{*a:}, as part of the regular pattern of quantitative alternation in denominative verbs built from the PIE suffix \pie{*y\eo}.\autocite[462]{Sihler_1995} This clears the way for the reconstructed form \hellenic{*\'ant\ae:s'}

\paragraph{VI.400: \greek{νήπιον}} The term \greek{νήπιον} is also quite problematic. Its etymology is quite unclear, and there's little information to build a satisfactory reconstruction with. The reconstruction here is keeping with the PIE form \pie{*\textsubring{n}-h}\textit{\textsubscript{2}}\pie{p-i\textsubarch{i}o}, as presented by Beekes.\autocite[νήπιος]{Beekes_2009}

\paragraph{VI.403: \greek{ἐρύετο Ἴλιον}} The re-insertion of the approximant \ipa{/w/} introduces a certain assonance to the piece. The strong repetitions add a cadence that, In my opinion, is quite beautiful. It is a new way of engaging with the piece, and I firmly believe more effort should be put into promoting and sharing these sorts of activities. Go ahead and read it out loud!

\begin{quote}
    \hellenic{wastuw\'anakt'}\textit{:} \hellenic{o\textsubarch{I}wos g\`ar wer\'ueto w\'\i lion \inipartvoiceless{\'e}ktO:r}
\end{quote}