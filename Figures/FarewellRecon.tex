% Book 6
\begin{Versi}[390]
\hellenic{\^E: \r{r}a gun\`\ae: tam\'\i\ae:, h\`o d' ap\'et't'uto d\'O:matos h\'ektO:r}\\
%ἦ ῥα γυνὴ ταμίη, ὃ δ' ἀπέσσυτο δώματος Ἕκτωρ\\
\hellenic{t\`\ae n aut\`\ae n hod\`on \'a\textsubarch{u}tis e\"uktim\'ena:s kat' agu\textsubarch{I}\'a:s}\\
%τὴν αὐτὴν ὁδὸν αὖτις ἐϋκτιμένας κατ' ἀγυιάς.\\
\hellenic{\'e\textsubarch{u}te p\'ula:s \'\i:ka:ne dierk\super{h}\'omenos m\'ega w\'astu}\\
%εὖτε πύλας ἵκανε διερχόμενος μέγα άστυ\\
\hellenic{skaiw\'as, t\'\ae:\textsubarch{I} ar' \'emelle dieks\'\i mena\textsubarch{I} ped\'\i onde}\\
%Σκαιάς, τῇ ἄρ' ἔμελλε διεξίμεναι πεδίονδε\\
\hellenic{\'ent\super{h}' \'alok\super{h}os pol\'udO:ros enant\'\i\ae: \greek{ἦλθε} t\super{h}\'ewo:sa}\\
%ἔνθ' ἄλοχος πολύδωρος ἐναντίη ἦλθε θέουσα\\ % Possibly
\hellenic{androm\'ak\super{h}\ae: t\super{h}ug\'atE:r megal\'E:toros E:et\'\i O:nos}\\
%Ἀνδρομάχη θυγάτηρ μεγαλήτορος Ἠετίωνος\\ % Possibly suffix -won, cf pion
\hellenic{E:et\'\i O:n h\`os \'ena\textsubarch{I}en hup\`o pl\'akO:\textsubarch{I} hu:l\ae:w\'et't'\ae:\textsubarch{I}}\\
%Ἠετίων ὃς ἔναιεν ὑπὸ Πλάκῳ ὑληέσσῃ\\ % Suffix -eis (-wents), cf haimatoweis
\hellenic{t\super{h}\'E:b\ae:\textsubarch{I} hupoplak\'\i\ae\textsubarch{I} \greek{κιλίκεσσ' ἄνδρεσσιν} wan\'at't'O:n}\textit{:}\\
%Θήβῃ Ὑποπλακίῃ Κιλίκεσσ' ἄνδρεσσιν ἀνάσσων:\\
\hellenic{t\^o: per d\`E: t\super{h}ug\'atE:r \'ek\super{h}et\super{h}' h\'ektori k\super{h}alkokorust\'\ae:\textsubarch{I}.}\\
%τοῦ περ δὴ θυγάτηρ ἔχεθ' Ἕκτορι χαλκοκορυστῇ.\\
\hellenic{h\'\ae: ho\textsubarch{I} \'epe:t' \greek{ἤντησ'}, h\'ama d' amp\super{h}\'\i polos k\'\i en aut\'\ae:\textsubarch{I}}\\
%ἥ οἱ ἔπειτ' ἤντησ', ἅμα δ' ἀμφίπολος κίεν αὐτῇ\\ % ἀμφίπολος < ἀμφίκϝολος -- kw -> p/t too early for PAI
\hellenic{p\'a:wida k\'olpO:\textsubarch{I} \'ek\super{h}o:s' atal\'ap\super{h}rona \greek{νήπιον} \'a\textsubarch{u}tO:s}\\
%παῖδ' ἐπὶ κόλπῳ ἔχουσ' ἀταλάφρονα νήπιον αὔτως\\ % -φρων< φρήν <- *gʷʰren-; νήπιος <- ν(ᾱ/η)ϝέπ-ι-ος
Ἑκτορίδην ἀγαπητὸν ἀλίγκιον ἀστέρι καλῷ,\\ % Haven't done this row yet
τόν ῥ' Ἕκτωρ καλέεσκε Σκαμάνδριον, αὐτὰρ οἱ ἄλλοι\\
Ἀστυάνακτ': οἶος γὰρ ἐρύετο Ἴλιον Ἕκτωρ.\\
ἤτοι ὃ μὲν μείδησεν ἰδὼν ἐς παῖδα σιωπῇ:\\
Ἀνδρομάχη δέ οἱ ἄγχι παρίστατο δάκρυ χέουσα,\\
ἔν τ' ἄρα οἱ φῦ χειρὶ ἔπος τ' ἔφατ' ἔκ τ' ὀνόμαζε:\\
δαιμόνιε φθίσει σε τὸ σὸν μένος, οὐδ' ἐλεαίρεις\\
παῖδά τε νηπίαχον καὶ ἔμ' ἄμμορον, ἣ τάχα χήρη\\
σεῦ ἔσομαι: τάχα γάρ σε κατακτανέουσιν Ἀχαιοὶ\\
πάντες ἐφορμηθέντες: ἐμοὶ δέ κε κέρδιον εἴη\\
σεῦ ἀφαμαρτούσῃ χθόνα δύμεναι: οὐ γὰρ ἔτ' ἄλλη\\
ἔσται θαλπωρὴ ἐπεὶ ἂν σύ γε πότμον ἐπίσπῃς\\ \ladd{\ldots}\\
\end{Versi}
\begin{Versi}[466]
%ὣς εἰπὼν
\ladd{\ldots} οὗ παιδὸς ὀρέξατο φαίδιμος Ἕκτωρ:\\
ἂψ δ' ὃ πάϊς πρὸς κόλπον ἐϋζώνοιο τιθήνης\\
ἐκλίνθη ἰάχων πατρὸς φίλου ὄψιν ἀτυχθεὶς\\
ταρβήσας χαλκόν τε ἰδὲ λόφον ἱππιοχαίτην,\\
δεινὸν ἀπ' ἀκροτάτης κόρυθος νεύοντα νοήσας.\\
ἐκ δ' ἐγέλασσε πατήρ τε φίλος καὶ πότνια μήτηρ:\\
αὐτίκ' ἀπὸ κρατὸς κόρυθ' εἵλετο φαίδιμος Ἕκτωρ,\\
καὶ τὴν μὲν κατέθηκεν ἐπὶ χθονὶ παμφανόωσαν:\\
αὐτὰρ ὅ γ' ὃν φίλον υἱὸν ἐπεὶ κύσε πῆλέ τε χερσὶν\\
εἶπεν ἐπευξάμενος Διΐ τ' ἄλλοισίν τε θεοῖσι:\\
Ζεῦ ἄλλοι τε θεοὶ δότε δὴ καὶ τόνδε γενέσθαι\\
παῖδ' ἐμὸν ὡς καὶ ἐγώ περ ἀριπρεπέα Τρώεσσιν,\\
ὧδε βίην τ' ἀγαθόν, καὶ Ἰλίου ἶφι ἀνάσσειν:\\
καί ποτέ τις εἴπῃσι πατρός δ' ὅ γε πολλὸν ἀμείνων\\
ἐκ πολέμου ἀνιόντα: φέροι δ' ἔναρα βροτόεντα\\
κτείνας δήϊον ἄνδρα, χαρείη δὲ φρένα μήτηρ.\\
\end{Versi}