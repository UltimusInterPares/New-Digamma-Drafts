\clearpage
\section{Cold Open}\label{sec:ColdOpen}
When we try to reconstruct Ancient Greek, we are wrong.  Reconstructive linguistics, like any other scientific process, is one of trial and error.  Philologists are constrained by history's slow erosion of knowledge: writing appearing just too late to capture a sound change, delicate scrolls being lost to unforgiving elements, humans destroying that which does not seem worth saving.  Our viewpoint of how people spoke and wrote in prehistoric times has been whittled away by every lost text into the smallest of apertures.  Our only means of understanding the past is oblique by necessity.  It is the process of looking at the holes left by historical changes and constructing the pieces that may have once fit there.  There are many such holes in our understanding of Ancient Greek, and one that fits well enough for me.

To be specific, what is missing is the quality of a phoneme: its actual specific pronunciation. There are, broadly speaking, two ways of describing a phoneme's quality: first by specifying its place and manner of articulation relative to the mouth, and second by identifying its distinctive features.

In consonants, the place of articulation describes where the tongue rests in the mouth during an utterance, such as between the teeth (``interdental'') or against the soft palate (``velar''). The manner of articulation describes the quality of airflow through the mouth during utterance. A stop, such as \ipa{/t/}, allows no airflow, while a fricative, such as \ipa{/f/}, allows a highly-turbulent flow.

Place is similar in vowels, but since they are more sonorant (that is, they create less turbulence in the vocal tract), they are described in terms of how high and front the tongue is during utterance. The vowel \ipa{/i/} is called a high front vowel, since the tongue is close to the roof of the mouth and extended towards the teeth, while \ipa{/A/} is a low back vowel, since the tongue is against the bottom of the mouth, and retracted towards the pharynx.

\begin{table}[htbp]
\rowcolors{2}{gray!25}{white}
\centering
\caption{Reconstructed Attic-Ionic Phonemic Consonants.}
\label{tab:Greek-Cons}
\resizebox{\textwidth}{!}{%
\begin{tabular}{@{}rcccccccccc@{}}
\toprule
 &
  \multicolumn{2}{c}{Labial} &
  Dental &
  \multicolumn{2}{c}{Alveolar} &
  Post-Alveolar &
  Palatal &
  \multicolumn{2}{c}{Velar} &
  Glottal \\ \midrule
Stop &
  \ipa{/p/} \greek{p} &
  \ipa{/b/} \greek{b} &
  \multicolumn{2}{c}{\ipa{/t/} \greek{t}} &
  \multicolumn{2}{c}{\ipa{/d/} \greek{d}} &
   &
  \ipa{/k/} \greek{k} &
  \ipa{/g/} \greek{g} &
   \\
Aspirated Stop &
  \ipa{/p\super{h}/} \greek{f} &
   &
  \multicolumn{2}{c}{\ipa{/t\super{h}/} \greek{j}} &
   &
   &
   &
  \ipa{/k\super{h}/} \greek{q} &
   &
   \\
Nasal &
   &
  \ipa{/m/} \greek{m} &
   &
   &
  \multicolumn{2}{c}{\ipa{/n/} \greek{n}} &
   &
   &
  \ipa{/N/} \greek{g($\nicefrac{\textrm{\greek{g}}}{\textrm{\greek{k}}}$)} &
   \\
Trill &
   &
   &
   &
   &
  \multicolumn{2}{c}{\ipa{/r/} \greek{r}} &
   &
   &
   &
   \\
Fricative &
   &
   &
   &
  \ipa{/\textsubbar{s}/} \greek{sv} &
  \ipa{/\textsubbar{z}(d)/} \greek{z} &
   &
   &
   &
   &
  \ipa{/h/} \greek{\<{v}} \\
Approximant &
   &
   &
   &
   &
   &
   &
  \textcolor{gray}{\ipa{/*j/} \greek{\j}} &
   &
   &
   \\
Lateral Approximant &
   &
   &
   &
   &
  \multicolumn{2}{c}{\ipa{/l/} \greek{l}} &
   &
   &
   &
   \\ \bottomrule
\end{tabular}%
}
\end{table}
\begin{table}[htbp]
\rowcolors{2}{gray!25}{white}
\centering
\resizebox{\textwidth}{!}{%
\begin{tabular}{@{}rrlrlrlccccrlrlcrrlrlrlrlccccrlrlrl@{}}
\toprule
\multicolumn{1}{l}{} &
  \multicolumn{14}{c}{Short Vowels} &
  \multicolumn{1}{l}{} &
  \multicolumn{1}{l}{} &
  \multicolumn{18}{c}{Long Vowels} \\ \midrule
 &
  \multicolumn{2}{c}{Front} &
  \multicolumn{1}{c}{} &
  \multicolumn{1}{c}{} &
  \multicolumn{1}{c}{} &
  \multicolumn{1}{c}{} &
   &
  \multicolumn{2}{c}{Central} &
   &
  \multicolumn{1}{c}{} &
  \multicolumn{1}{c}{} &
  \multicolumn{2}{c}{Back} &
   &
   &
  \multicolumn{2}{c}{Front} &
  \multicolumn{1}{c}{} &
  \multicolumn{1}{c}{} &
  \multicolumn{1}{c}{} &
  \multicolumn{1}{c}{} &
  \multicolumn{1}{c}{} &
  \multicolumn{1}{c}{} &
   &
  \multicolumn{2}{c}{Central} &
   &
  \multicolumn{1}{c}{} &
  \multicolumn{1}{c}{} &
  \multicolumn{1}{c}{} &
  \multicolumn{1}{c}{} &
  \multicolumn{2}{c}{Back} \\ \cmidrule(lr){2-15} \cmidrule(l){18-35} 
High &
  \ipa{i} &
  \greek{ῐ} &
   &
   &
   &
   &
   &
   &
   &
   &
   &
   &
  \ipa{u} &
  \greek{ῠ} &
   &
  High &
  \ipa{i:} &
  \greek{ῑ} &
   &
   &
   &
   &
   &
   &
   &
   &
   &
   &
   &
   &
   &
   &
  \ipa{u:} &
  \greek{ῡ} \\
Mid-High &
   &
   &
  \ipa{e} &
  \greek{e} &
   &
   &
   &
   &
   &
   &
  \ipa{o} &
  \greek{o} &
   &
   &
   &
  Mid-High &
   &
   &
  \ipa{e:} &
  \greek{ει} &
   &
   &
   &
   &
   &
   &
   &
   &
   &
   &
  \ipa{o:} &
  \greek{ου} &
   &
   \\
Mid-Low &
   &
   &
   &
   &
   &
   &
   &
   &
   &
   &
   &
   &
   &
   &
   &
  Mid-Low &
   &
   &
   &
   &
  \ipa{E:} &
  \greek{η} &
   &
   &
   &
   &
   &
   &
  \ipa{O:} &
  \greek{ω} &
   &
   &
   &
   \\
Low &
   &
   &
   &
   &
  \ipa{a} &
  \greek{ᾰ} &
   &
   &
   &
   &
   &
   &
   &
   &
   &
  Low &
   &
   &
   &
   &
   &
   &
  \ipa{a:} &
  \greek{ᾱ} &
   &
   &
   &
   &
   &
   &
   &
   &
   &
   \\ \bottomrule
\end{tabular}%
}
\caption{Attic-Ionic Phonemic Vowels -- Before Attic shifted \ipa{/u(:)/} to \ipa{/y(:)}}
\label{tab:Greek-Vowels}
\end{table}

Distinctive features are individual components of pronunciation, starting as vague as whether a phoneme is either a consonant [+consonant] or vowel [+vowel], and ending as specific as whether a phoneme is pronounced by obstructing airflow with the sides of the tongue [+laminal] or the tip [+apical]. Features can be written in-line, as shown here, or in a matrix, which is the typical method when writing formal sound change rules.