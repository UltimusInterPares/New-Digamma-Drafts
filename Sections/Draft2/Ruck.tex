\section{R\"uckverwandlung}\label{sec:Ruck}

\subsection{\rxout{Plausibility}}\label{subsec:Plaus2}

\subsection{Chronology}\label{subsec:Chronology}
\dpara{R\"uck 1}
The beginning of \w-elision can also be confidently placed after
the onset of the Attic R\"uckverwandlung, which saw the newly-created
vowel \ipa{\ae:} revert back to a long \ipa{/a:/} before a rho \ortho{\textel{ρ}},
epsilon \ortho{\textel{ε}}, or iota \ortho{\textel{ι}}; this is the process which
created the final segment \textel{-έᾱ} in \textel{νέᾱ}. \textel{Κόρη}, however, seems to
violate this rule with its final segment \textel{-ρη}. Given that the AI
quantitative alternation implies the CH root \lroot{kor\w-}, with a root-final
consonant \w\ between the rho \ortho{\textel{ρ}} and what was, at the time,
the feminine ending *-\={a}. If the consonant \w\ had been elided before the AI
vowel raising, the sequence would have appeared as CH \iform[L]{k\'or\w-\=a} $\to$
\iform[L]{k\'or-\=a} $\to$ \iform[L]{k\'or-\ash} $\to$ \iform[G]{κόρᾱ}. Since the consonant \w\ blocked the
process of R\"uckverwandlung, the correct relative sequence can be
confidently reconstructed as *kor\w-\=a $\to$ *kor\w-\ash\ $\to$ kor\w-\ipa{\=E}
$\to$ \textel{κόρη} 

\dpara{R\"uck 2}