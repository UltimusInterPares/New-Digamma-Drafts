\section{R\"uckverwandlung}\label{sec:Ruck}

\subsection{\rxout{Plausibility}}\label{subsec:Plaus2}

\subsection{Chronology}\label{subsec:Chronology}
\dpara{R\"uck 1}\label{para:Ruck1}
The beginning of \w-elision can also be confidently placed after
the onset of the Attic R\"uckverwandlung, which saw the newly-created
vowel \ipa{\ae:} revert back to a long \ipa{/a:/} before a rho \ortho{\textel{ρ}},
epsilon \ortho{\textel{ε}}, or iota \ortho{\textel{ι}}; this is the process which
created the final segment \textel{-έᾱ} in \textel{νέᾱ}. \textel{Κόρη}, however, seems to
violate this rule with its final segment \textel{-ρη}. Given that the AI
quantitative alternation implies the CH root \lroot{kor\w-}, with a root-final
consonant \w\ between the rho \ortho{\textel{ρ}} and what was, at the time,
the feminine ending *-\={a}. If the consonant \w\ had been elided before the AI
vowel raising, the sequence would have appeared as CH \iform[L]{k\'or\w-\=a} $\to$
\iform[L]{k\'or-\=a} $\to$ \iform[L]{k\'or-\ash} $\to$ \iform[G]{κόρᾱ}. Since the consonant \w\ blocked the
process of R\"uckverwandlung, the correct relative sequence can be
confidently reconstructed as *kor\w-\=a $\to$ *kor\w-\ash\ $\to$ kor\w-\ipa{\=E}
$\to$ \textel{κόρη} 

\dpara{R\"uck 2}\label{para:Ruck2}
R\"uckverwandlung, however, could not possibly have happened in one
singular instance. \textel{Νέᾱ}, as mentioned above, still underwent
R\"uckverwandlung, despite the intervocalic hiatus indicating the
original root \groot{νέ\w-}. \W-elision, then, must have elided some time
after R\"uckverwandlung following a rho \ortho{\textel{ρ}}, but before R\"uckverwandlung
following the front vowels epsilon \ortho{\textel{ε}} and iota \ortho{\textel{ι}},
giving the process two distinct phases.

\dpara{Contr.}\label{para:Contr}
Both of these processes occurred after the onset of vowel contractions.
\textel{Νέᾱ}, again as mentioned above, shows intervocalic hiatus in the segment
\textel{-έᾱ}, which is consistent with elision \w-elision in other lemmas.
S- and \textel{\yod}-elision, however, both allow contractions between vowels, as seen in
in the genitive singular \textel{γένους} $\gets$ *g\'enesos,\autocite[172]{Sihler_1995}
and the first person singular \textel{νικῶ} $\gets$ nik\'aj\=o.
If contractions had occurred after \w-elision, terms like \textel{ἐννέα} $\gets$ CH *enn\'e\w\textel{\shwa}
would read \iform[G]{ἐννῆ}