\section{Lengthening}\label{sec:Lengthening}

\subsection{Plausibility}\label{subsec:Plausibility}
\dpara{QA}
Quantitative alternations between Attic and Ionic point to the presence of the historical
consonant \w. \bookref{Iliad}{1}{474} shows \textel{κᾱλός} with a long alpha \ortho{\textel{ᾱ}} where
Attic has a short alpha \ortho{\textel{ᾰ}}. This could be another case of a short vowel becoming long by position,
but \edit{\ldots}{something about lots of other vowels being long too}. Forms like
\textel{κόρη}, \textel{ξένος}, and \textel{ὅρος} all have Ionic counterparts with a long vowel:
\textel{κούρη}, \textel{ξεῖνος}, and \textel{οὖρος} respectively. So while root-initial \w-clusters caused
a preceding short vowel to scan as long across root boundaries, root-medial clusters caused a
short vowel to become long by nature in Ionic, while having no affect in Attic.
In the cases of epsilon \ortho{\textel{ε}} and omicron \ortho{\textel{ο}}, there is no reason to assume that the vowels are long by position
as that metrical tactic is never marked with a change in a vowel's spelling, only by the
presence of multiple following consonants.

\dpara{S-E?}
A possible explanation for these long vowels is s-elision, triggering a process called
compensatory lengthening. Terms such as \textel{εἰμί} $\gets$ \CH\ \ipa{*esm\'\i} and \textel{σελήνη} $\gets$ \ipa{*sel\'{\u{a}}sn\=a}