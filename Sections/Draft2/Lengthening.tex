\section{Lengthening}\label{sec:Lengthening}

\subsection{Plausibility}\label{subsec:Plausibility}
\dpara{QA}
Quantitative alternations between Attic and Ionic point to the presence of the historical
consonant \w. \bookref{Iliad}{1}{474} shows \textel{κᾱλός} with a long alpha \ortho{\textel{ᾱ}} where
Attic has a short alpha \ortho{\textel{ᾰ}}. This could be another case of a short vowel
becoming long by position,
but \edit{\ldots}{something about lots of other vowels being long too}. Forms like
\textel{κόρη}, \textel{ξένος}, and \textel{ὅρος} all have Ionic counterparts with a long vowel:
\textel{κούρη}, \textel{ξεῖνος}, and \textel{οὖρος} respectively. So while root-initial \w-clusters caused
a preceding short vowel to scan as long across root boundaries, root-medial clusters caused a
short vowel to become long by nature in Ionic, while having no affect in Attic.
In the cases of epsilon \ortho{\textel{ε}} and omicron \ortho{\textel{ο}}, there is no reason to
assume that the vowels are long by position
as that metrical tactic is never marked with a change in a vowel's spelling, only by the
presence of multiple following consonants.

\subsection{S-Elision}\label{subsec:SElision}
\dpara{S-E 1}
A possible explanation for these long vowels is s-elision, triggering a process called
compensatory lengthening. Terms such as \textel{εἰμί} $\gets$ \CH\ \ipa{*esm\'\i} and \textel{σελήνη} $\gets$
CH \ipa{*sel\'{\u{a}}sn\=a} show that an original *s was elided, which triggered the lengthening of
the preceding vowel.\footnote{Strictly speaking, this is not accurate. The \IE\ *s had become an *h in
in most positions at the beginning of the \CH\ era. However, since the process starts with an IE *s,
and since the particular steps leading towards complete elision are not entirely relevant to this paper,
I have opted to not mark the \CH\ *h. I may come to regret this at some point, but here we are.}
This happens through a process called Mora Preservation, whereby the individual mor\ae,
the relative units of metrical length, are moved in order to maintain the specific length of a word
after an elision.

% add Mora Preservation

\dpara{S-E 2}
S-elision, however, does not provide a satisfactory explanation in the case of \textel{καλός}.
This process occurred late in, or shortly after, the \CH\ period, as evidenced by the differing
reflexes across the Greek dialects; and the \AI\ family actively participated in
compensatory lengthening after s-elision. CH \iform[L]{kals\'os} would have resulted
in AI \iform[G]{κᾱλός}, with a long alpha in both dialects.\autocite[229]{Sihler_1995}

\subsection{J-Elision}\label{subsec:JElision}
\dpara{J-E 1}
Another explanation may be some lengthening in response to \textel{\yod}-elision. Terms like
\textel{τείνω} $\gets$ CH *t\'e\v{n}\v{n}\=o $\gets$ IE t\'enjoh\textsubscript{2}, from a group of
verbs with the so-called `iota present',\footnote{that is, the \IE\ present with the infix -\textel{\yod}\eo-.}
shows a long vowel \phonem{e:} \ortho{\textel{ει}} where the root \groot{τεν-} has a short \phonem{e}
\ortho{\textel{ε}}. This could indicate that the elision of \textel{\yod}\ lead to another
compensatory lengthening.

\dpara{J-E 2}
This is, however, impossible for a few reasons. First, the IE form -\textel{\yod}\eo- created
numerous verb forms in Greek, such as verbs in \textel{-έω}, which typically contracts to \textel{-ῶ}
in Attic, and which shows a short epsilon \ortho{\textel{ε}} where compensatory lengthening
would have created a long vowel \phonet{e:} \ortho{\textel{ει}}. Second, the verb has
an alternate reduplicated form \textel{τιτάινω} which, while agreeing with the root \groot{τ\eaν-},
shows a diphthong \phonet{aj} \ortho{αι} where compensatory lengthening would have created a long
monophthong \iform[L]{\phonet{a:}} \iform[G]{\ortho{ᾱ}}. This suggests that an intervocallic \textel{\yod}\ 
was lost
without triggering compensatory lengthening, and that the long \phonet{e:} \ortho{ει} in \textel{τείνω}
was the reflex of a genuine diphthong, created by the same metathetic process which made the diphthong
in \textel{τιταίνω}. This process was part of a greater system called the Greek
Palatalization,\autocite[197]{Sihler_1995} which, crucially for \textel{κ\B{\M{α}}λός}, changed the segment
\textel{*λ\yod} into \textel{λλ}, as in \textel{ἄλλος} $\gets$ CH *\'al\textel{\yod}os. The result of
\CH\ \iform[L]{k\u{a}l\textel{\yod}\'os} would have been \iform[G]{κᾰλλός}, with a short alpha
\ortho{\textel{ᾰ}} and a geminate lambda \ortho{\textel{λλ}}.

% Table of relevant, non-spoiler palatalization, please!

\subsection{\W-Elision}\label{subsec:WElision}
\dpara{\W-E}
The remaining option is compensatory lengthening induced by \w-elision. Terms like \textel{κόρη},
\textel{μόνος}, and \textel{ξένος} all show alternations in situations where a *Cs or *C\textel{\yod} segment
would have caused lengthening or a diphthong. Since the only other phonemes to elide without conditining
were *s and *\textel{\yod}, the last remaining option is the consonant \w, which was already shown to have
elided in all positions, as indicated by historical scansion. This has two major implications: first,
precedent implies that the sound was some sort of lenis consonant, either a fricative or approximant;
second, the differences between Attic and Ionic reflexes indicate that the consonant \w\ was lost
after the two dialects had split, at a much later date than s- and \textel{\yod}-elision.
\edit{Notably, the elision must have occurred after the \AI\ vowel shift, which can be used to
establish a relative chronology of sound changes.}{This can be incorporated into the following paragraph.}

% Need to distinguish between VwC/V and VCw