\section{Homeric}\label{sec:Homeric}
\missingpara{Intro}

\subsection{Elision}\label{subsec:Elision}
\dpara{Elision}
\edit{\ldots}{Transition: something about ``historic scansion.''}
Frequently, that takes the form
of the vowels failing to elide across word boundaries, which may indicate some now-absent
phoneme, present during the composition of the version of the Epics transmitted by our
scribal tradition, but lost some time before the transcription and subsequent editing that gave
us the po\"ems in their modern form. \bookref{Iliad}{15}{214} gives the example
\textel{``Ἡφαίστοιο ἄνακτος''},\autocite[XV.214]{Iliad_1999}
without any contraction between the genitive ending
\textel{-οιο} and the following alpha.
This indicates that the missing sound was a consonant (hereafter \w) and not
a vowel (hereafter \vowel), since a missing vowel would have still allowed elision,
giving the sequence \iform[G]{Ἡφαίστοιο \vowelάνακτος} $\to$ \iform[G]{Ἡφαίστοι' \vowelάνακτος}
$\to$ \iform[G]{Ἡφαίστοι' ἄνακτος}.

\textcolor{red}{\w\ features 1}

\subsection{Meter}\label{subsec:Meter}
\dpara{Scan 1}
\edit{\ldots}{Something about inconsistent representation due to later editing}
Another feature of this historic scansion involves vowels
that are phonologically short scanning as though they were long.
\bookref{Iliad}{5}{308} shows the segment \textel{``ἀπὸ ῥίνον''} scans with a long omicron
\ortho{\textel{\M{ο}}},\autocite[V.306]{Iliad_1999}
Even though in typical orthography, the omicron \ortho{\textel{ο}} exclusively encodes the
short vowel \phonem{o}.
However, standard metrical practice allowed a short vowel to scan as long, with the title
\textit{long by position},
when followed by two or more consonants, whether in the same word or across word boundaries. This implies
that the rho \ortho{\textel{ρ}} in \textel{ῥίνον} was originally part of a consonant
cluster, either \textel{\wρ} or \textel{ρ\w}. A cluster in this position would cause the
preceding omicron \ortho{\textel{ο}} to scan as long, and explain the apparent discrepancy
in the historic scansion.

\dpara{Scan 2}
Outstanding long vowels also point towards the same process occurring word-initially.
Scansion of \bookref{Iliad}{1}{33} shows the word \textel{\Asm{ε}δεισεν} with a short
epsilon \ortho{\textel{ε}} which, as with \textel{ἀπὸ}, scans as long by
position.\autocite[I.33]{Iliad_1999}
This indicates that the root-initial delta \ortho{\textel{δ}} was similarly part of a consonant cluster,
either \textel{\wδ} or \textel{δ\w}, which was able to affect scansion word-medially.
\edit{\ldots}{Something about quantitative alternation}

\subsection{Intent}\label{subsec:Intent}
\missingpara{Intent}