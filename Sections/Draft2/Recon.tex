\section{Reconstructions}\label{sec:Recon}

\subsection{Roots}\label{subsec:Roots}
The main process used thus far to find the consonant *\w\ between vowels,
that is, searching for and analyzing intervocalic hiatus, can be
partially automated. For this purpose, I have written three short scripts
in a programming language called R, which help to gather and parse
appropriate evidence.

The first script, called HomeR, is a text miner: a program which reads a
corpus of text, manipulates it, and provides the requested output. At
the present moment, HomeR is directed to read every book of the Iliad,
track each line number (both relative to the book number and to the piece
as a whole), then strip every accent it finds aside from the diaereses.
It then takes every word from its line and places it in a new row of
a table, along with the relevant line and book numbers. In this way, every
single word of the Iliad -- all 111,861 of them, in the edition used here --
is prepared for analysis.

HomeR has two companions, the first of which is named ThRax, for the
Hellenistic grammarian Dionysios Thrax, or \textel{Διονύσιος ὁ Θρᾷξ}.
ThRax is a Sound Change Applier (SCA), whose sole function is to receive
a hypothetical reconstructed form of a word, and estimate a descendant
form. This relies on the premise implied by the relative chronology
above, that sound changes happen in a rigid order. This means that sound
changes can be described in a program and systematically applied, allowing
a reconstructed word form to be passed through a list of changes and
compared to the attested word order in order to test the veracity of the
reconstruction.

The second companion is named GoRgias, after the ancient sophist. GoRgias
is responsible for filtering the output from HomeR in order to build a table of
potentially useful terms, then guessing at a hypothetical reconstruction,
which it passes along to ThRax. After the reconstructions are run through
the relevant sound changes, it and its hypothetical descendant are added
to the table so that forms can be compared.

These three scripts have been directed to read through the Iliad, then
filter the resulting table down to one particular scene, Hector's farewell
to Andromache, specifically \bookref{IL}{6}{390-412, 466-482}. % Break para to avoid back-to-back fig.s

\textcolor{red}{[Add the farewell scene]}

This passage
is then filtered again to find any instances of intervocalic hiatus, which
are then separated by inserting the consonant *\w. This happens in three
phases. First, the consonant is placed between two monophthongs, such as
\ipa{/e.o/} or \ipa{/a.E:/}. Second, it is placed between a diphthong and
a monophthong, such as \ipa{/a\textsubarch{I}.E:/} or \ipa{/y\textsubarch{I}.o/}.
The instructions to GoRgias are worded in such a way that this step will also
capture hiatus between diphthongs, such as \ipa{/a\textsubarch{I}.o\textsubarch{I}/}.
Lastly, the consonant is added between a monophthong and a diphthong, such
as \ipa{/e.o\textsubarch{I}/}.

Running GoRgias with these parameters provides a small number of roots
which may have once held the consonant *\w.

\textcolor{red}{[Example of an output from GoRgias]}

It should be noted that these scripts are somewhat greedy, and will regularly
over-capture. For instance, the denominative suffixes \textel{-ιος} $\gets$
\IE\ *-ios, \textel{-ιᾰ} $\gets$ IE *-i\textsubring{h}\textsubscript{2}, and
\textel{-ίη} $\gets$ IE *-ieh\textsubscript{2} are all regularly captured and
reconstructed as \CH\ \iform[L]{-i\w os}, \iform[L]{-i\w a}, and \iform[L]{-i\w\=a}.
This is, however, currently preferable to under-capturing, as some word forms
would be inappropriately excluded, such as \textel{ἰάχω} and \textel{Ἀχαιοί}
$\gets$ CH *\w i\w ak\textsuperscript{h}\=o and
*Ak\textsuperscript{h}a\textsubarch{\textsci}o\textsubarch{\textsci}. Following
through to confirm captured words returns a list of ten roots in the passage
which may have
had the consonant \w\ at the end of the Common-Attic-Ionic period.

\textcolor{red}{[Roots Table]}
%\begin{table}[htbp]
\centering
\begin{tabular}{@{}lllllll@{}}
\toprule
Attested &
  Reconstructed &
  PAI Root &
   &
  Attested &
  Reconstructed &
  PAI Root \\ \midrule
\greek{Σκαιάς} &
  *\hellenic{ska\textsubarch{I}\.{c}as} &
  *\lroot{\hellenic{ska\textsubarch{I}\.{c}-}} &
   &
  \greek{πάϊς} &
  *\hellenic{pa\.{c}is} &
  *\lroot{\hellenic{pa\.{c}-}} \\
\greek{θέουσα} &
  *\hellenic{t\super{h}e\.{c}o:sa} &
  *\lroot{\hellenic{t\super{h}e\.{c}-}}  &
   &
  \greek{ἰάχων} &
  *\hellenic{\.{c}i\.{c}ak\super{h}O:n} &
  *\lroot{\hellenic{\.{c}ak-}} \\
 %%%%%%%%%%%%%%%%%%%
\greek{χέουσα} &
  *\hellenic{k\super{h}e\.{c}o:sa} &
  *\lroot{\hellenic{k\super{h}e\.{c}-}} &
 %%%%%%%%%%%%%%%%%
   &
  \greek{υἱόν} &
  *\hellenic{hy\textsubarch{I}\.{c}on} &
  *\lroot{\hellenic{hy\textsubarch{I}\.{c}-}} \\
\greek{ἐλεαίρες} &
   *\hellenic{ele\.{c}a\textsubarch{I}res} &
   *\lroot{\hellenic{ele\.{c}-}} &
   &
  \greek{ἐρύετο} &
  *\hellenic{\.{c}eryeto} &
  *\lroot{\hellenic{\.{c}eru-}} \\
\greek{Ἀχαιοί} &
  *\hellenic{ak\super{h}a\textsubarch{I}\.{c}o\textsubarch{I}} &
  *\lroot{\hellenic{ak\super{h}a\textsubarch{I}\.{c}-}} &
   &
   &%\greek{θεοὶ} &
   &%*\hellenic{t\super{h}e\.{c}o\textsubarch{I}} &
   \\ \bottomrule
\end{tabular}
\caption[Reconstructed Common-Attic-Ionic Roots]{Reconstructed Common-Attic-Ionic Roots}
\label{tab:recon-roots}
\end{table}

\subsection{Suffixes}\label{subsec:Suffixes}
Together, the scripts also identify a number of suffixes with the consonant *\w.
This includes some more common suffixes, such as \textel{-αρ, -ων}, and,
in the passage specifically, \textel{-εις}.
Some, such as \textel{-ος}, can be confidently distinguished from similar forms --
here, the o-stem nominative \textel{-ος} -- by their extraneous hiatus,
but others require more investigation.

The suffixes \textel{-αρ} and \textel{-ων}, found in \textel{στέαρ} and \textel{πίων},
are both confirmed by examining other derivations from shared roots. \textel{Στέαρ}
shares a root with both \textel{στάσις} and \textel{ἵστημι}, which provide
the root \groot{στ\ea-}, with the alternations in vowel quality being a result
of IE ablaut; \footnote{Cf. Smyth 35-56} while \textel{Πίων} shares its root with
\textel{πιμέλη}. Given that the forms \textel{στέαρ} and \textel{πίων} show signs
of a missing consonant, while their cognate derivatives do not, it can be
confidently argued that the consonant was a part of the suffix, not the roots.
Both forms disqualify an intervocalic *-s- or *-j-, given the lack of any
discernible effect on the preceding vowel following elision.
No compensatory lengthening occurred, disqualifying s-elision, and no
diphthongs were formed, disqualifying j-elision. The remaining options, then,
are the forms \textel{στέαρ} $\gets$ CAI *st$\nicefrac{e}{a}$-\w ar and
*pi-\w \=on. 

The base form of *-\w\=on, however, must have had a short vowel.
The suffix shows quantitative alternation between cases, with the
Genitive Singular \textel{-ονος} and the Dative Singular \textel{-ονι},
suggesting that the omicron \spell{\textel{ο}} is lengthened when in the
word-final position in much the same way as the final vowel in
\textel{πάτηρ} and \textel{ἄνηρ}. This gives the suffix' base form
\textel{-on-}, reconstructed as PAI *-\w\u{o}n-.

\textcolor{red}{\textel{-ων} endings here}

The suffix \textel{-εις} can be reconstructed with an initial *\w.
It is attested three times in the passage, in the words \textel{ὑληέσσῃ},
\textel{Τρώεσσιν}, and \textel{βροτόεντα}.
It can be reasonably expected that the suffix begin with a consonant
given the construction of the word \textel{βροτ-ό-εντα}, attested with
an omicron \spell{\textel{ο}} inserted as a connecting vowel, which only
occurred when the addition of an affix to a stem would have created a
consonant cluster. The omicron, then, must have been added to separate
the root-final tau \spell{\textel{τ}} and whatever consonant followed.
If that consonant had been *-s- or *-j-, then its elision would have
allowed for contraction between the connecting omicron \spell{\textel{ο}}{o}
and the suffix-initial epsilon \spell{\textel{ε}} giving the form
\iform[G]{βροῦντα}. The initial consonant, then, must have again been the
hypothetical consonant \w.

Much like the suffix \textel{-ων}, the Nominative forms of \textel{-εις}
obfuscate the base form. Looking at the Genitive, Dative, and Accusative
forms -- \textel{-εντος}, \textel{-εντι}, and \textel{-εντα}, respectively --
show the base form \textel{-εντ-}. The Nominative form can be explained
as the result of two successive consonant contractions, giving the
process \textel{-εντς} $\to$ \textel{-εινς} $\to$ \textel{-εις}, with
the long epsilon \spell{\textel{ει}}{e:}
arising through compensatory
lengthening. This indicates the CAI form *-\w ents.

\dpara{(e)u} GoRgias also demonstrates a connection between the two
u-type agent suffixes, \textel{-υς} and \textel{-ευς}. The script, when
instructed to examine the chosen passage, captures \textel{υἱόν}, the
Accusative Singular of \textel{υἱύς}; when instructed to examine the
Iliad as a whole, it captures such terms as \textel{ἡδύς},
\textel{Ἀχιλλεύς}, and \textel{βασιλεύς}. The declension of the -type
endings show a strong concordance in their endings (though
their exact form varies somewhat to satisfy metrical requirements).
This indicates an etymological relationship between the two, where
\textel{-ευς}, with a present epsilon \spell{\textel{ε}}{e}, is the
thematic form of \textel{-υς}, which lacks any such vowel. 

\textcolor{red}{\textel{-(ε)υς} endings here}

\dpara{BRING IT HOME}
These endings are also noteworthy for what they indicate about the
consonant *\w: the position of the intervocalic hiatus in these
endings can indicate the specific quality, the actual pronunciation,
of the hypothetical consonant.
Take the Dative Singular ending, variously attested as \textel{-ηϊ}
or \textel{-εϊ}. The hiatus stands exactly where the Nominative
singular shows the upsilon \spell{\textel{υ}}{y}, which apparently
disappeared in declension -- the dative, otherwise, would read as
\iform[G]{-ηυϊ} or \iform[G]{-ευι}. It has already been determined
that the consonant was likely either an approximant or a fricative,
and given that it was related to the upsilon, and that some vowels
have a relative approximant, specifically in the form of a relative
semivowel, the pronunciation of the upsilon
can potentially identify the consonant \w's specific pronunciation.

However, given the historical nature of this relationship, it is not
useful to examine the pronunciation of upsilon \ortho{υ} as it was
pronounced in the classical era. At this point, both Attic and
Ionic speakers pronounced this as the front round vowel \ipa{/y/}
(cf. Fr \ortho{\textrm{u}} and De \ortho{\textrm{\"u}}, however it
was fronted some time between 700 and 400 \textsc{b.c.e.}, moving
from its original pronunciation as a back round vowel \ipa{/u/} (cf.
Fr \ortho{\textrm{ou}} and De \ortho{\textrm{u}}).\autocite[529]{malikouti-drachman_bortone_2015}
In this instance, then, the vowel upsilon \ortho{υ} needs to be
analyzed as that back round vowel. 

\textcolor{red}{[Semivowel relationship chart]}
%\begin{table}[htbp]
\centering
\caption{Greek Phonemes with Relative Semivowels}
\label{tab:semivowels}
\begin{tabular}{lll}
Greek                       & Vowel   & Semivowel \\
\textel{ι}                  & \ipa{i} & \ipa{j}   \\
\multirow{2}{*}{\textel{υ}} & \ipa{y} & \ipa{4}   \\
                            & \ipa{u} &          
\end{tabular}
\end{table}

With this in mind, the behavior of the upsilon \ortho{υ} indicates
that the historic consonant likely was pronounced as \ipa{/w/},
the sound of the Gnglish letter double-u \ortho{\textrm{w}}. 
The pronunciation of the two sounds are near-identical, excepting
for differences in quantity -- that is, the vowel \ipa{/u/} is typically
held longer than the semivowel \ipa{/w/}. In fact, the specific features
of these sounds -- literally the physical methods used in their pronunciation,
such has tongue position and tension -- have a near perfect correlation,
except that \ipa{/w/} is labeled as a consonant and \ipa{/u/} a vowel.
This would explain the behavior of the upsilon \ortho{υ} in u-type
endings, with the short, intervocalic /u/ being re-bracketed, being
understood not as the end of the prior syllable but the beginning of the
latter, and then re-analyzed as a semivowel, which was subsequently elided.
The dative, then, progressed from the Common-Attic-Ionic
*-$\nicefrac{e}{\bar{e}}$u.i $\to$ *-$\nicefrac{e}{\bar{e}}$.ui $\to$
*-$\nicefrac{e}{\bar{e}}$.wi $\to$ Homeric
-\textel{$\nicefrac{\textrm{\textel{ε}}}{\textrm{\textel{η}}}$ϊ}

\textcolor{red}{[Input final reconstruction!!!?!?!1`!]}