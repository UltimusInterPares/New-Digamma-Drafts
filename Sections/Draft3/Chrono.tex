%\section{R\"uckverwandlung}\label{sec:Ruck}
\section{Chronology}\label{sec:Chrono}

\subsection{Elision}

The differing results of Compensatory Lengthening following H- and \W-Elision make it necessary to determine the general order of sound changes between the Common-Hellenic language and the Attic-Ionic dialect family. Chronologies of this sort build on one fundamental premise: sound changes happen in a set order. This need not be the exact same order between languages, but it is stable throughout a language's life. This is because they happen over time, one-by-one, and changing the order would necessitate changing the language as spoken at some point in the past -- something which I cannot do, at any rate. Given this, it is possible to place sound changes into a relative chronology, a list irrespective of the actual date of change. Compensatory Lengthening provides a convenient starting point.

As mentioned above, the eta \ortho{\greek{η}} in H \greek{δῆνος} and the epsilon-iota \ortho{\greek{ει}} in H \greek{ξείνος} imply that H-Elision occurred before \W-Elision. The key here is the height of the long vowels. The short \hellenic{/*e/} (English \english{a} in \english{ate}) in the root \lroot{*denh-} is not only lengthened, but also lowered to a long \hellenic{/*E:/} (E \english{e} in \english{end}). There are three possible explanations to this. First, the short \hellenic{/*e/} could have itself been a short \hellenic{/*E/}. Unfortunately, this would raise the question of why \greek{ξείνος} shows a different result from Compensatory Lengthening. If it were a short \hellenic{/*E/}, then the natural reflex would have been \iform[G]{ξήνος}. Second, the short vowel \hellenic{/*e/} may have naturally evolved to \hellenic{/*E:/} in a single motion when undergoing compensatory lengthening. The issue, again, is that this leaves no room for \greek{ξείνος}: if \hellenic{/*e/} lengthened directly to \hellenic{/*E:/}, then the root \lroot{*ksen\w-} should evolve into \iform[G]{ξήνος}. The third option is that \greek{δῆνος} was the result of a two step process. The short \hellenic{/*e/} in the root \hellenic{*denh-} first became a long \hellenic{/*e:/}, which was then lowered to \hellenic{/E:/} as part of an unconditioned shift. The shift from CH \hellenic{/*e:/} from PIE \pie{*e} or \pie{*eh}\textit{\textsubscript{1}}, to G \hellenic{/E:/} is a known feature. The long eta \ortho{\greek{η}} in the negative particle \greek{μή} is the result of CH \hellenic{*m\=e}, and in \greek{ἡμι-} from CH \hellenic{*s\=emi-}.\autocite[51]{Smyth_2013} The long epsilon-iota in \greek{ξείνος}, then, must have occurred after this general lowering. 

H- and J-Elision, on the other hand, must sit adjacent to one another in the chronology. Attic and Homeric share the majority of reflexes,\autocite[198]{Smyth_2013} for example, PIE \pie{*-gy-} and \pie{*-dy-} become A/H \greek{ζ} as in \greek{ἅζομαι} from \pie{*hag-y\eeoo-} and \greek{ἐλπίζω} from \pie{*elpid-yoh}\textit{\textsubscript{2}}.\autocite[200]{Smyth_2013} The consistent reflexes imply that J-Elision occurred before the appearance of distinctly Attic and Homeric features. Following that mindset, the different reflexes of \W-Elision in Attic and Homeric then imply that, while both dialects lost the hypothetical consonant *\w, they did so after the dialect family split, allowing for differences in derived terms.

\subsection{R\"uckverwandlung}
Dialectical differences also show that \W-Elision occurred before the completion of the Attic-Ionic Vowel Shift, placing it near to the very beginning of the split in the Attic-Ionic family. Where Homeric shows the form \greek{κούρη}, with a long omicron-upsilon \ortho{\greek{ου}} and a long final eta \ortho{\greek{η}}, Attic shows \greek{κόρη} with, surprisingly enough, yet another long eta. At some point during the late Common-Attic-Ionic era, long \hellenic{/*a:/} raised to a long \hellenic{/*\ae:/}. After the dialects split, this vowel continued to raise to \hellenic{/E:/}, but it in Attic it first underwent an extra process called \german{Rückverwandlung}, or \english{Reversal}. This sound change states that any instance of \hellenic{/*\ae:/} reverts back to \hellenic{/*a:/} if it follows an \hellenic{/*r/}, \hellenic{/*e/}, or \hellenic{/*i/}. For example, where Homeric shows \hellenic{καρδίη}\ ``heart'', Attic shows \hellenic{καρδίᾱ}. The alternation between a short omicron \ortho{\greek{ο}} and a long omicron-upsilon \ortho{\greek{ου}} indicate that the root once ended in a cluster, and the quality of the long vowel indicates that the cluster was \hellenic{/*r\w/}. This consonant must have remained until after \german{Rückverwandlung} began, since its presence would block the lowering of \hellenic{/*r\ae:/} back to \hellenic{/*ra:/}. \german{Rückverwandlung} did, however, occur where there was once an intervocalic consonant \hellenic{*\w}, as in A \greek{ἐννέᾱ}\ ``nine'' $\gets$ CH \hellenic{*enne\w\greek{\shwa}}.\autocite[ἐννέα]{Beekes_2009} This confirms that \german{Rückverwandlung} occurred in two distinct phases: first after \hellenic{/*r/} and later after \hellenic{/*e/} and \hellenic{/*i/}, with \W-Elision occurring at some point between the two.

%\noindent\textcolor{red}{Input chronology}
\begin{table}[htbp]
\centering

\begin{tabular}{lll|l|l}
    &                              &                                                         & \ipa{*kór\.{c}a:}   & \ipa{*né\.{c}a:}   \\
%(1) & Vowel Contractions          &\phon{eo, ea}{ō, ā}                                      &                     &                    \\
(1) &H-Elision                    &\phon{\ipa{h}}{$\textcolor{gray}{\emptyset}$}             &                     &                    \\
(2) &J-Elision                    &\phon{\ipa{j}}{$\textcolor{gray}{\emptyset}$}             &                     &                    \\
(3) &Long-Mid Vowel Lowering      &\phon{\ipa{e:, o:}}{\ipa{E:, O:}}                         &                     &                    \\
(4) &Attic Vowel Raising 1        &\phon{\ipa{a:}}{\ipa{\ae:}}                               & \ipa{*kór\.{c}\ae:} & \ipa{*né\.{c}\ae:} \\
(5) &Rückverwandlung 1            &\phonc{\ipa{\ae:}}{\ipa{a:}}{\ipa{r}\phold}               &                     &                    \\
(6) &Intervocalic \.{C}-Elision   &\phonc{\ipa{\w}}{$\textcolor{gray}{\emptyset}$}{V\phold V}&                     & \ipa{*né\ae:}      \\
(7) &Postconsonantal \.{C}-Elision&\phonc{\ipa{\w}}{$\textcolor{gray}{\emptyset}$}{C\phold}  & \ipa{*kór\ae:}      &                    \\
(8) &Rückverwandlung 2            &\phonc{\ipa{\ae:}}{\ipa{a:}}{\ipa{e(:), i(:)}\phold}      &                     & \ipa{néa:}         \\
(9) &Attic Vowel Raising 2        &\phon{\ipa{\ae:}}{\ipa{E:}}                               & \ipa{kórE:}         &                    \\
    &                             &                                                          & \textel{κόρη}       & \textel{νέᾱ}
    
\end{tabular}
\caption{Relative Chronology of Attic Sound Changes}
\label{tab:Chronology}
\end{table}