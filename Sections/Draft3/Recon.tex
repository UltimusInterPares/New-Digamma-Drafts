\section{Reconstructions}\label{sec:Recon}
The one-by-one nature of sound changes leaves room for automation. Regular changes mean allow for a computer to search for evidence of a process, guess at reconstructions, and test the veracity of its own guesses.  To this end, I have written three short scripts in a programming language called R, which currently search for instances of the hypothetical consonant *\w\ between vowels.

The first script is named HomeR. It is a text miner, which is to say that it reads, manipulates, and assists in the analysis of written text. Currently, it reads the entire text of the Iliad, one book at a time. HomeR then places each line into a table, and labels the lines with their Global Line Number, which is the line's position in the epic; their Book Number, which is the book from which a line is retrieved; and the Relative Line Number, which is the line's position in that book. HomeR then tokenizes the text, which is to say it splits each line into distinct tokens. A token is a textual variable, which can be defined as one of multiple levels, including lines of text, words, and individual letters. In this instance, HomeR defines a token as a word, and so places each word into its own row along with the requisite book and line numbers.

%\noindent\textcolor{red}{Input HomeR output}
\begin{table}[htbp]
\centering
\begin{tabular}{@{}rllll@{}}
\toprule
  & GlobalLine & Book & RelativeLine & Word              \\ \cmidrule(l){2-5} 
1 & 1          & 1    & 1            & \greek{μηνιν}     \\
2 & 1          & 1    & 1            & \greek{αειδε}     \\
3 & 1          & 1    & 1            & \greek{θεα}       \\
4 & 1          & 1    & 1            & \greek{πηληϊαδεω} \\
5 & 1          & 1    & 1            & \greek{αχιληος}   \\ \bottomrule
\end{tabular}
\caption{Iliad I.1 as Output by HomeR}
\label{tab:HomeR-Example}
\end{table}

The second script is named ThRax, after the Hellenistic grammarian Dionysios Thrax. It is a sound change applier, a program which holds the details of any phonological process, including the phoneme(s) to change, the constraints of the change, and the result. ThRax takes a word or series of words as an input, either from a list or the column of a table, which then pass through each programmed sound change in order, and then return as newer derivations. 

ThRax cannot assert the validity of a reconstruction on its own, since more than one possible reconstruction may give the same result. Both \hellenic{*kal\w os} and \iform[L]{kalhos}, for example, would both return \greek{κᾱλός}, a valid derivation for Homeric Greek. Prior knowledge of the different reflexes of \W-Elision in Attic and Homeric Greek is necessary to identify \iform[L]{kalhos} as an incorrect reconstruction. Some reconstructions, however, can be disqualified with a surface-level analysis of ThRax' output. The incorrect reconstruction \iform[L]{kaljos} would return \iform[L]{κᾰλλός}, which is disqualifiable at a glance.

The third and final script is named GoRgias, after the Athenian sophist and philosopher. GoRgias is a management tool, and the primary means of interfacing with HomeR and ThRax. It has four components. First, it reads the output from HomeR and sets the scope of the text to be analyzed. Second, it reads through that scope and sets the parameters to search for. Third, it transcribes the text in question into the International Phonetic Alphabet. Lastly, it guesses at a reconstruction of whatever terms match the search parameters, and sends them through ThRax. The filtered results from HomeR, the guessed reconstructions, and the results from ThRax are then bundled into a table for further analysis. Given that ThRax relies on manually defined sound change rules, the user can easily alter what point in time the scripts should reconstruct to. Any word form could be treated as older or newer depending on how far back in the chronology ThRax is programmed for. If ThRax were given every change back to the start of the Common-Hellenic period, then the input from GoRgias would essentially be treated as a Common-Hellenic form, since it would pass through all of the changes restricted to forms from that time period.  Since this paper has focused on reconstructing the hypothetical consonant *\w\ to its latest possible days, and since the elision of the consonant *\w\ predates the split between the Attic and Ionic dialects, the scripts will work to reconstruct the consonant as it was in the last days of the Common-Attic-Ionic period. 

\marginnote{Rough transition. (Minimalist?)} When searching for roots and affixes, GoRgias is instructed to reduce the scope of analysis to one scene from book VI, namely  Hector's farewell to Andromache and his prayer for Astyanax, in \bookref{Iliad}{6}{390-412; 476-481}.. It then searches through the passage and filters the results to any instances of intervocallic hiatus. The consonant *\w\ is then inserted between vowels in three phases. First, the consonant is placed between two monophthongs, such as \ipa{/e.o/} or \ipa{/a.E:/}. Second, it is placed between a diphthong and a monophthong, such as \ipa{/a\textsubarch{I}.E:/} or \ipa{/y\textsubarch{I}.o/}. The instructions to GoRgias are worded in such a way that this step will also capture hiatus between diphthongs, such as \ipa{/a\textsubarch{I}.o\textsubarch{I}/}. Lastly, the consonant is added between a monophthong and a diphthong, such as \ipa{/e.o\textsubarch{I}/}. 

%\noindent\textcolor{red}{Input GoRgias example}
\begin{table}[htbp]
\centering
\resizebox{\textwidth}{!}{%
\begin{tabular}{@{}rllllll@{}}
\toprule
  & GlobalLine & Book & RelativeLine & Word                & \W List                   & applysoundchange(\W List) \\ \cmidrule(l){2-7} 
1 & 3792       & 6    & 390          & \greek{ταμιη}       & \hellenic{tami\w ɛ:}      & \hellenic{tamiɛ:}         \\
2 & 3793       & 6    & 391          & \greek{εϋκτιμενας}  & \hellenic{e\w üktimenas}  & \hellenic{eüktimenas}     \\
3 & 3793       & 6    & 391          & \greek{αγυιας}      & \hellenic{agyĭ\w as}      & \hellenic{agyĭas}         \\
4 & 3794       & 6    & 392          & \greek{διερχομενος} & \hellenic{di\w erxomenos} & \hellenic{dierxomenos}    \\
5 & 3795       & 6    & 393          & \greek{σκαιας}      & \hellenic{skaĭ\w as}      & \hellenic{kaĭas}          \\ \bottomrule
\end{tabular}
}
\caption{Example Output of GoRgias}
\label{tab:GoRgias-Example}
\end{table}

Note that these scripts are greedy, and will generally over-capture. For instance, the denominal suffixes \greek{-ιος} from PIE \pie{*-i-os}, \greek{-ιᾰ} $\gets$ PIE \pie{*-i-h}\textit{\textsubscript{2}}, and \greek{-ίη} $\gets$ PIE \pie{-\'\i-eh}\textit{\textsubscript{2}} are all consistently captured and reconstructed as Common-Hellenic \iform[L]{-i\w os}, \iform[L]{-i\w\u{a}}, and \iform[L]{-i\w a:}. This is preferable to under-capturing, as some forms may be inappropriately excluded, such as \greek{ἰάχω} $\gets$ CH \hellenic{*(\w)i\w ak\super{h}o:} and \greek{Ἀχαιοί} $\gets$ CH  \hellenic{*ak\super{h}a\textsubarch{I}\w o\textsubarch{I}}. Analysis of GoRgias' findings indicates a list of ten roots in the passage which may have had the consonant *\w\ at the end of the Common-Attic-Ionic period.

%\noindent\textcolor{red}{Input roots list}
\begin{table}[htbp]
\centering
\begin{tabular}{@{}lllllll@{}}
\toprule
Attested &
  Reconstructed &
  PAI Root &
   &
  Attested &
  Reconstructed &
  PAI Root \\ \midrule
\greek{Σκαιάς} &
  *\hellenic{ska\textsubarch{I}\.{c}as} &
  *\lroot{\hellenic{ska\textsubarch{I}\.{c}-}} &
   &
  \greek{πάϊς} &
  *\hellenic{pa\.{c}is} &
  *\lroot{\hellenic{pa\.{c}-}} \\
\greek{θέουσα} &
  *\hellenic{t\super{h}e\.{c}o:sa} &
  *\lroot{\hellenic{t\super{h}e\.{c}-}}  &
   &
  \greek{ἰάχων} &
  *\hellenic{\.{c}i\.{c}ak\super{h}O:n} &
  *\lroot{\hellenic{\.{c}ak-}} \\
 %%%%%%%%%%%%%%%%%%%
\greek{χέουσα} &
  *\hellenic{k\super{h}e\.{c}o:sa} &
  *\lroot{\hellenic{k\super{h}e\.{c}-}} &
 %%%%%%%%%%%%%%%%%
   &
  \greek{υἱόν} &
  *\hellenic{hy\textsubarch{I}\.{c}on} &
  *\lroot{\hellenic{hy\textsubarch{I}\.{c}-}} \\
\greek{ἐλεαίρες} &
   *\hellenic{ele\.{c}a\textsubarch{I}res} &
   *\lroot{\hellenic{ele\.{c}-}} &
   &
  \greek{ἐρύετο} &
  *\hellenic{\.{c}eryeto} &
  *\lroot{\hellenic{\.{c}eru-}} \\
\greek{Ἀχαιοί} &
  *\hellenic{ak\super{h}a\textsubarch{I}\.{c}o\textsubarch{I}} &
  *\lroot{\hellenic{ak\super{h}a\textsubarch{I}\.{c}-}} &
   &
   &%\greek{θεοὶ} &
   &%*\hellenic{t\super{h}e\.{c}o\textsubarch{I}} &
   \\ \bottomrule
\end{tabular}
\caption[Reconstructed Common-Attic-Ionic Roots]{Reconstructed Common-Attic-Ionic Roots}
\label{tab:recon-roots}
\end{table}

\subsection{Suffixes}

%\noindent\textcolor{red}{Love yourself and write a para for -wos. Here, I'll even get you started!}
%for example, -ος in κεραός "horned" can be differentiated from the typical nominative -ος in δρόμος "race". 

Together, the scripts also identify a number of suffixes with the consonant *\w.
This includes some more common suffixes, such as \greek{-ος, -αρ, -ων}, and,
in the passage specifically, \greek{-εις}.

The adjectival suffix \greek{-ος} is distinguishable from the nominal \greek{-ος} in two key ways. First, in vowel-stem terms, the suffix-initial omicron \ortho{\greek{ο}} conspicuously does not contract with the stem-final vowel, such as in \greek{κεραός} ``horned''. Second, in consonant-stem terms, Homeric shows long vowels corresponding to Attic short vowels, such as in A \greek{ὅλος} and H \greek{οὖλος}, or A \greek{ξένος} and H \greek{ξεῖνος}. There is no lowering of mid-high long vowels, implying compensatory lengthening from the loss of *\w, and giving the Common-Attic-Ionic suffix \hellenic{*-\w os}

The suffixes \greek{-αρ} and \greek{-ων}, found in \greek{στέαρ} and \greek{πίων},
are both confirmed by examining other derivations from shared roots. \greek{Στέαρ}
shares a root with both \greek{στάσις} and \greek{ἵστημι}, which provide
the root \groot{στ\ea-}, with the alternations in vowel quality being a result
of IE ablaut; \footnote{Cf. Smyth 35-56} while \greek{πίων} shares its root with
\greek{πιμέλη}. Given that the forms \greek{στέαρ} and \greek{πίων} show signs
of a missing consonant, while their cognate derivatives do not, it can be
confidently argued that the consonant was a part of the suffix, not the roots.
Both forms disqualify an intervocalic \hellenic{*-h-} or \hellenic{*-j-}, given the lack of any
discernible effect on the preceding vowel following elision.
No compensatory lengthening occurred, disqualifying H-Elision, and no
diphthongs were formed, disqualifying J-Elision. The remaining options, then,
are the forms \greek{στέαρ} $\gets$ CAI \hellenic{*st$\nicefrac{e}{a}$-\w ar} and
\hellenic{*pi-\w O:n}. 

The base form of \hellenic{*-\w O:n}, however, must have had a short vowel.
The suffix shows quantitative alternation between cases, with the
Genitive Singular \greek{-ονος} and the Dative Singular \greek{-ονι},
suggesting that the omicron \ortho{\greek{ο}} is lengthened when in the
word-final position in much the same way as the final vowel in
\greek{πάτηρ} and \greek{ἄνηρ}. This gives the suffix' base form
\greek{-on-}, reconstructed as PAI \hellenic{*-\w\u{o}n-}.

%\textcolor{red}{\textel{-ων} endings here}
\begin{table}[htbp]
	\centering
        \begin{tabular}{@{}lccc@{}}
        \toprule
             & \multicolumn{2}{c}{M \& F}       & N           \\ \cmidrule(l){2-4} 
        Nom. & \multicolumn{2}{c}{\textel{ων}}  & \textel{ον} \\
        Gen. & \multicolumn{3}{c}{ονος}                       \\
        Dat. & \multicolumn{3}{c}{ονι}                        \\
        Acc. & \multicolumn{2}{c}{\textel{ονα}} & \textel{ον} \\
        Voc. & \multicolumn{3}{c}{ον}                         \\ \bottomrule
        \end{tabular}
	\caption{\textel{-ων} Singular Case Endings}
	\label{tab:ων-Endings}
\end{table}

The suffix \greek{-εις} can be reconstructed with an initial *\w.
It is attested twice times in the passage, in the words \greek{ὑληέσσῃ} and \greek{βροτόεντα}.
It can be reasonably expected that the suffix begin with a consonant
given the construction of the word \greek{βροτ-ό-εντα}, attested with
an omicron \ortho{\greek{ο}} inserted as a connecting vowel, which only
occurred when the addition of an affix to a stem would have created a
consonant cluster. The omicron, then, must have been added to separate
the root-final tau \ortho{\greek{τ}} and whatever consonant followed.
If that consonant had been \hellenic{*-h-} or \hellenic{*-j-}, then its elision would have
allowed for contraction between the connecting omicron \ortho{\greek{ο}}
and the suffix-initial epsilon \ortho{\greek{ε}} giving the form
\iform[G]{βροῦντα}. The initial consonant, then, must have again been the
hypothetical consonant \w.

Much like the suffix \greek{-ων}, the Nominative forms of \greek{-εις}
obfuscate the base form. Looking at the Genitive, Dative, and Accusative
forms -- \greek{-εντος}, \greek{-εντι}, and \greek{-εντα}, respectively --
show the base form \greek{-εντ-}. The Nominative form can be explained
as the result of two successive consonant contractions, giving the
process \greek{-εντς} $\to$ \greek{-ενς} $\to$ \greek{-εις}, with
the long epsilon \ortho{\greek{ει}}
arising through compensatory
lengthening. This indicates the CAI form \hellenic{*-\w ents}.

\marginnote{Can you tell this was just slapped in here?}
Lastly, and in a similar manner to \hellenic{*-\w os}, hiatus also indicates the consonant *\w\ in the suffix \greek{-ως}, as in \greek{Τρώς} ``A Trojan''. However, unlike the other suffixes found in the passage, he Genitive and Dative singular forms \greek{Τρωός} and \greek{Τρῶϊ}, indicate a suffix-final consonant. In the case of \greek{Τρώς}, other forms were built with secondary suffixes (suffixes appended onto a prior suffix), such as in \greek{Τρωϊκός} ``Trojan'', \greek{Τρώϊος} ``Trojan man'', and \greek{Τρώϊα} ``Trojan women''. This strategy is also responsible for the form \greek{ἡρωΐνη} $\gets$ CAI \hellenic{hE:rO:\w\'\i n\ae}

\begin{table}[htbp]
\centering
\begin{tabular}{@{}llll@{}}
\toprule
Attested     & Reconstructed       &                          &                                     \\ \midrule
             & M                   & F                        & N                                   \\ \cmidrule(l){2-4} 
\greek{-ος } & \hellenic{*-\w o-s} & \hellenic{*-\w e-}       & \hellenic{*-\w o-n}                 \\
\greek{-αρ}  &                     &                          & \hellenic{*-\w a$\nicefrac{r}{t}$-} \\
\greek{-ων}  & \multicolumn{2}{l}{\hellenic{*-\w\u={o}n-}}    & \hellenic{*-\w\u{o}n-}              \\
\greek{-εις} & \hellenic{*-\w \u{e}nt-} & \hellenic{*-\w et't'-\u={a}} & \hellenic{*-\w $\nicefrac{en}{ent-}$} \\
\greek{-ως}  & \hellenic{*\=o\w-s} & \hellenic{*\=o\w-in-\=a} &                                     \\ \bottomrule
\end{tabular}
\caption{Reconstructed Common-Attic-Ionic Suffixes}
\label{tab:recon-suffixes}
\end{table}

%\dpara{(e)u}
GoRgias also demonstrates a connection between the two
u-type agent suffixes, \greek{-υς} and \greek{-ευς}. The script, when
instructed to examine the chosen passage, captures \greek{υἱόν}, the
Accusative Singular of \greek{υἱύς}; when instructed to examine the
Iliad as a whole, it captures such terms as \greek{ἡδύς},
\greek{Ἀχιλλεύς}, and \greek{βασιλεύς}. The declension of the u-type
endings show a strong concordance in their endings (though
their exact form varies somewhat to satisfy metrical requirements).
This indicates an etymological relationship between the two, where
\greek{-ευς}, with an epsilon \ortho{\greek{ε}}, is the
thematic form of \greek{-υς}, which lacks any such vowel. 

%\textcolor{red}{\greek{-(ε)υς} endings here}
\begin{table}[htbp]
	\centering
	\begin{tabular}{@{}lcc@{}}
		\toprule
		     & \greek{-υ-} & \greek{-ευ-}       \\ \cmidrule(l){2-3}
		Nom. & \textel{υς} & \textel{ευς}       \\
		Gen. & \multicolumn{2}{c}{\textel{εος}} \\
		Dat. & \multicolumn{2}{c}{\textel{εϊ}}  \\
		Acc. & \textel{υν} & \textel{εα}        \\
		Voc. & \textel{υ}  & \textel{ευ}        \\
		\bottomrule
	\end{tabular}
	\caption{\textel{-υ}- and \textel{-ευ}-Type Singular Endings}
	\label{tab:U-Endings}
\end{table}

%\dpara{BRING IT HOME}
These endings are also noteworthy for what they indicate about the
consonant *\w: the position of the intervocalic hiatus in these
endings can indicate the specific quality, the actual pronunciation,
of the hypothetical consonant.
Take the Dative Singular ending, variously attested as \greek{-ηϊ}
or \greek{-εϊ}. The hiatus stands exactly where the Nominative
singular shows the upsilon \ortho{\greek{υ}}, which apparently
disappeared in declension -- the dative, otherwise, would read as
\iform[G]{-ηυϊ} or \iform[G]{-ευι}. It has already been determined
that the consonant was likely either an approximant or a fricative,
and given that it was related to the upsilon, and that some vowels
have a relative approximant, specifically in the form of a relative
semivowel, the pronunciation of the upsilon
can potentially identify the consonant \w's specific pronunciation.

However, given the historical nature of this relationship, it is not
useful to examine the pronunciation of upsilon \ortho{υ} as it was
pronounced in the classical era. At this point, both Attic and
Ionic speakers pronounced this as the front round vowel \ipa{/y/},\footnote{Cf. Fr \french{u} and De \german{\"u}} however it
was fronted some time between 700 and 400 \textsc{b.c.e.}, moving
from its original pronunciation as a back round vowel \ipa{/u/}.\autocite[529]{malikouti-drachman_bortone_2015}
In this instance, then, the vowel upsilon \ortho{υ} needs to be
analyzed as that back round vowel. 

%\textcolor{red}{[Semivowel relationship chart]}
\begin{table}[htbp]
\centering
\caption{Greek Phonemes with Relative Semivowels}
\label{tab:semivowels}
\begin{tabular}{lll}
Greek                       & Vowel   & Semivowel \\
\textel{ι}                  & \ipa{i} & \ipa{j}   \\
\multirow{2}{*}{\textel{υ}} & \ipa{y} & \ipa{4}   \\
                            & \ipa{u} &          
\end{tabular}
\end{table}

With this in mind, the behavior of the upsilon \ortho{υ} indicates
that the historic consonant likely was pronounced as \ipa{/w/},
the sound of the English letter double-u \ortho{\english{w}}. 
The pronunciation of the two sounds are near-identical, excepting
for differences in quantity. That is, the vowel \ipa{/u/} is typically
held longer than the semivowel \ipa{/w/}. Even the specific features
of these sounds have a near perfect correlation,
except that \ipa{/w/} is labeled as a consonant and \ipa{/u/} a vowel.
This would explain the behavior of the upsilon \ortho{\greek{υ}} in u-type
endings, with the short, intervocalic \ipa{/u/} being re-bracketed, being
understood not as the end of the prior syllable but the beginning of the
latter, and then re-analyzed as a semivowel, which was subsequently elided.
The dative, then, progressed from the Common-Attic-Ionic
\hellenic{*-$\nicefrac{e}{\bar{e}}$\textsubarch{u}.i} $\to$
\hellenic{*-$\nicefrac{e}{\bar{e}}$.\textsubarch{u}i} $=$
\hellenic{*-$\nicefrac{e}{\bar{e}}$.wi} $\to$ Homeric
-\greek{$\nicefrac{\textrm{\greek{ε}}}{\textrm{\greek{η}}}$ϊ}

\begin{figure}
    \centering
        \ipa{w} \\
        \phonfeat{
        $+$ Sonorant \\
        $+$ Back \\
        $+$ High \\
        $+$ Labial \\
        $+$ Continuant \\
        $+$ Voiced
    }
    \caption{Featurs of \ipa{w}}
    \label{fig:FeaturesIII}
\end{figure}