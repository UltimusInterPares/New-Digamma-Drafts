\section{Lengthening}\label{sec:Lengthening}

\subsection{Identifying CL}
\dpara{QA}\dnote{2 ¶s?} While the presence of the historic consonant \greek{*\w} effected the scansion of historic phrases, its loss introduced discernible effects on a preceding vowel in the same root. Epic Greek shows \greek{κούρη}, \greek{ξείνος}, \greek{οὖρος}, and \greek{κᾱλός} where Attic Greek shows \greek{κόρη}, \greek{ξένος}, \greek{ὅρος}, and \greek{κᾰλός}. These are distinguishable from vowels made long by position given that the Epic vowels show their length with spelling. That is, whereas \greek{ἀπό} and \greek{ἔδεισεν} do not mark length on the omicron or epsilon, \greek{κούρη} and \greek{οὖρος} mark length with a digraph \ortho{\greek{ου}}, and \greek{ξείνος} with \ortho{\greek{ει}}. Note that these digraphs do not represent the sequences \greek{ο + υ} and \greek{ε + ι}, but rather the long vowels \greek{\M{ο}} and \greek{\M{ε}}. The process responsible for these long vowels is called compensatory lengthening. Every phoneme in Greek can be measured in units of duration called mor\ae: short vowels and single consonants are said to have one mora, long vowels and diphthongs have two, and consonant clusters have two or more. When a consonant cluster with \greek{*\w} is reduced, the single mora from \greek{*\w} is transferred to a preceding vowel, as long as the two share a root. This process is significant in identifying the quality of \greek{*\w}, that is, its specific pronunciation. Since there is a known input, a short vowel and a consonant cluster with \greek{*\w},  a known condition, both are in the same root, and a known output, a long vowel and a single consonant, this process is comparable to other known cases of elision in Greek.

\noindent\textcolor{red}{Input Mora Preservation}

\subsection{H-Elision}
\dpara{HE I} The first possible option is H-Elision. As the Proto-Indo-European language developed into the Common-Hellenic language, the consonant \pie{*s} often shifted to the consonant \pie{*h}. This occurred at the beginning of a word, before a vowel (unless following a voiceless stop), after a vowel (unless at the end of a word or followed by a voiceless stop), and in a word-initial consonant cluster with either a nasal or an approximant.\autocite[168-172]{Smyth_2013} \edit{That is, the sequences \pie{*sm}, \pie{*sn}, \pie{*sl}, and \pie{*sr} became \hellenic{*mh}, \hellenic{*nh}, \hellenic{*lh}, and \hellenic{*rh}; for example, PIE \pie{*srud\super{h}mos} became CH \hellenic{*rhut\super{h}mos}, with a voiceless \hellenic{*/\textsubring{r}/}.\autocite[ῥυθμός]{Beekes_2009}}{Necessary?} The consonant \hellenic{*h} disappeared in Greek unless at the beginning of a word and, when part of a consonant cluster, this triggered compensatory lengthening. For example, Common-Hellenic \hellenic{*selahna} became Attic \greek{σελήνη}.\autocite[\textel{σελήνη}]{Beekes_2009}

\dpara{HE II} However, this does not explain the vowel alternations between the Attic and Homeric forms above. Attic and Homeric both agree on vowel lengths in the roots \groot{δην-} in A \greek{δήνεα} and H \greek{δῆνος} $\gets$ CH *\lroot{denh-},\autocite[\textel{δήνεα}]{Beekes_2009} and \groot{σελην-} in A and H \greek{σελήνη} $\gets$ CH *\lroot{selahn-}, as above. if H-Elision were responsible for the long vowels in Homeric \greek{ξείνος} and \greek{κούρη}, then the vowels in Attic would have matched. Furthermore, the vowel qualities are incorrect for compensatory lengthening after H-Elision. Homeric \greek{δῆνος} shows a long eta \ortho{\textel{η}}, cf. French \french{\`e}, while \greek{ξείνος} shows a long epsilon-iota digraph \ortho{\textel{ει}}, cf. F \french{\'e}. This means that H-Elision occurred before the long vowel \hellenic{*\=e} in CH lowered to \greek{η}, while the process which created the long \greek{ει} occurred after.

\subsection{J-Elision}
The last possible option is J-Elision. The semivowel \pie{*j}, cf. English \ortho{y}, itself very common in PIE, elided after another consonant by the end of the Common-Hellenic period.\autocite[196]{Smyth_2013} The loss of this often created long vowels of its own, for example Attic \greek{τείνω} $\gets$ PIE \pie{*t\'enjoh\textsubscript{2}} shows a long \ortho{\greek{-ειν-}} derived from an original  \pie{*-enj-}.\autocite[τείνω]{Beekes_2009} In this respect, the result of J-Elision compares to the result of H-Elision and Compensatory Lengthening.

Yet this analysis is misleading: while the input and output resemble those of H-Elision, the paths taken by the two processes are very different. The first sign is the quality of the long vowel created after J-elision. Given that this process occurred at a similar time to H-Elision, the vowel created by Compensatory Lengthening would have been \iform[G]{η}, as in H \greek{δῆνος} above, giving the form \iform[G]{τήνω}. Worse yet, the root of \greek{τείνω} shows frequent vowel alternations, with attested forms including \greek{τείνω}, \greek{τάνυται},\autocite[τάνυται]{Beekes_2009} and \greek{τιταίνω}.\autocite[τιταίνω]{Beekes_2009} This alternation suggests a different source of the iota \ortho{\textel{ι}} in \greek{τιταίνω}, and by extension \greek{τείνω}, since lengthening of a short \pie{*ᾰ} from PIE \pie{*titanjoh}\textit{\textsubscript{2}} would have resulted in a long \iform[G]{ᾱ}. The culprit here is a sound change called metathesis (G \greek{μετάθεισις}), wherein phonemes or syllables trade places in a word. In English, this is the process responsible for creating \english{aks} as an alternate form  of ask \english{ask}. In Greek, this specific instance of metathesis involved the semivowel \hellenic{*j} trading positions with,  among others, \hellenic{*n}, \hellenic{*r}, and \hellenic{*h}. In the instances of \greek{τείνω} and \greek{τιταίνω}, this change took place through a medial form: Proto-Indo-European \pie{*-\vowel nj-} $\to$ CH \hellenic{*-\vowel\~n\~n-} $\to$ A and H \greek{-\vowelιν-} (CH \hellenic{*\~n} being identical to Spanish \spanish{\~n}).

With all this in mind, it is impossible to assert that \w\ be either \hellenic{*s} or \hellenic{*j}, but these comparisons do provide insight to the quality of \w. During the transition from the Common-Hellenic language to Attic and Homeric, the only three consonants to elide without conditioning were \hellenic{*h}, \hellenic{*j}, and \w. That is to say, any other sound that disappeared from a word did so as the result of some other grammatical or phonological process. For example, A \greek{πᾶς} shows the root \groot{πάντ-}. While the cluster \ortho{\greek{-ντ-}} does elide, it is a conditioned change: dental consonants, and clusters of dentals, elide when they come into contact with a sigma \ortho{\greek{σv}}. In this case, the final sigma \ortho{\greek{-s}} is the nominative ending.  So in Attic, the process is as follows: \greek{πάντ-ς} to \greek{πάν-ς} $\to$ \greek{πᾶ-ς}. Regular H- and J-Elision, on the other hand, are unconditioned changes: \hellenic{*h} and \hellenic{*j} elide in most every case, with only small concessions to the surrounding phonemes. Given that most most elisions involve \hellenic{*h}, and \hellenic{*j}, a comparison to \w\ can be drawn by examining the overlap between these phonemes. Namely, \hellenic{*h} and \hellenic{*j} are not stop consonants: neither involve the complete blocking of airflow through the mouth like \ipa{/p/} and \ipa{/k/}. The phoneme \hellenic{/*h/} is a fricative, pronounced by restricting the glottis (the ``vocal folds'') to create turbulent airflow. The phoneme \hellenic{/*j/} is an approximant, pronounced by creating some turbulence between the tongue and the roof of the mouth, though not enough to qualify as a fricative. While not confirmatory, this gives precedence to analyze the missing consonant \w\ as either a fricative or approximant.

\noindent\textcolor{red}{input [+-fric / +- approx]}