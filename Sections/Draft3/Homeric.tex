\section{Homeric}\label{sec:Homeric}

To start reconstructing, we need to begin gathering evidence, and find something to compare that evidence against. For evidence, we can look to the Iliad of Homer, some of the oldest surviving writing in Greek. The Greek of Homer, called Homeric or Epic Greek, survives as a subset of the Ionic dialect: while a rhapsode would have performed in whatever dialect was spoken by their audience, and would have even altered the course of the piece to fit the tastes of the listeners. Instead of learning the entire piece, these poets memorized pre-built lines, which they used as a foundation to improvise the rest of the epic. These are the oldest lines of the Iliad, surviving from well before the first scribes recorded the poem. As such, they show irregularities stemming from historic scansion of static phrases.

%\dpara{Apocope}
Such irregularities often take the form of the vowels not eliding across word boundaries. In Greek, like in French and Italian, a word-terminal vowel will often elide before a word-initial vowel. That is to say, a vowel at the end of a word will drop if the following word begins with another vowel. Compare the elision in Greek  \greek{μεθ' ἡμῖν} ``with us'', from \greek{μετὰ ἡμῖν}, to French  \french{J’ai sommeil} ``I am tired'' $\gets$ \french{Je ai sommeil}. \bookref{Iliad}{15}{214}, however, gives the example \greek{Ἡφαίστοιο ἄνακτος} without any apocope, or vowel elision, between the genitive ending \greek{-οιο} and the following alpha.\autocite[XV.214]{Iliad_1999} This implies that there was some phoneme, some distinct sound, was present at the start of the word \greek{ἄνακτος}. This phoneme was a consonant, which will be written as *\w\ until the exact sound can be determined, rather than a vowel, *\vowel, as a vowel would have still allowed elision. The sequence with a vowel would read \iform[G]{Ἡφαίσοιο \vowelάνακτος} to \iform[G]{Ἡφαίστοι' \vowelάνακτος} $\to$ \iform[G]{Ἡφαίστοι' ἄνακτος}.\footnote{\edit{Note that an asterisk before any form means that it is unattested, which is to say that it does not survive in the written record. As such, that form is hypothetical, based on the current understanding of a language -- in this instance, the language is Greek, but the system is convention. A superscript \textit{x} before a form means that it is expected to be incorrect. Either it is a hypothetical form that does not fit with the current understanding of a language, or it is a form that contradicts extant records.}{We're putting these in a footnote for now. Don't get to comfy with that.}}

\begin{figure}
    \centering
        \w \\
        \phonfeat{
        $+$ Consonant
    }
    \caption{Currently Known Feature of \W}
    \label{fig:FeaturesI}
\end{figure}

%\dpara{LBP}\dnote{2 ¶s?}
Another type of historic scansion involves short vowels scanning as long.  
The Iliad, like most epics, uses a system called dactylic hexameter. Each line divides into six units, hence the \textit{hex-} from Greek \greek{ἕξ}\ ``six''. Each unit divides into either three syllables -- one long followed by two shorts -- or two syllables -- two longs in a row.
%\rxout{These resemble the shape of a finger,}
%\rxout{\textit{dactyl-} coming from Greek \greek{δάκτυλος} ``finger''. Long}\\
%\rxout{ syllables represent the first bone of the finger, while the short syllables represent the second}\\
%\rxout{and third bones.}
In any dactylic meter, a short vowel followed by one or no consonants creates a short syllable. A short vowel followed by two or more consonants creates a long syllable, regardless of if those consonants are in the same word, or across a word boundary. A long vowel creates a long syllable in any position. \bookref{Iliad}{5}{308}  has the phrase \greek{ἀπὸ ῥίνον}, with the omicron \ortho{\textel{ο}} scanning as long, even though the letter only represents a short vowel.\autocite[V.803]{Iliad_1999} In this instance, the omicron should scan as short \ortho{\textel{\B{ο}}}, being a short vowel only followed by one consonant. \edit{Given that this omicron scans as long \ortho{\textel{ο}}, the following rho \ortho{\textel{ρ}} must be an element of a consonant cluster.}{Make very clear that this is long by position} \edit{The apocopic forms \greek{ἀπ'} and \greek{ἀφ'} disqualify the form \greek{ἀπό\w},}{point out instances of apocopic απ´ and that ϝ would block elision} leaving only two possible positions for the hypothetical consonant: either before the rho, \ortho{\textel{*\wρ}}, or after it, \ortho{\textel{*ρ\w}}.

%\dpara{LBP II} \dnote{Cut?} \rxout{Affixes can also point to a similar process happening within a word boundary.}\\
%\rxout{Scansion of \bookref{Iliad}{1}{33} shows the word \greek{\Asm{ε}δεισεν},\autocite[I.33]{Iliad_1999} with a word-initial epsilon \greek{ἐ-} counting as a}\\
%\rxout{long vowel. In the aorist tense, or the ``simple past'' tense, the verb takes a prefix called the}\\
%\rxout{epsilon augment. As with the letter omicron, epsilon encodes a short vowel, implying that}\\
%\rxout{another consonant is missing. The possible clusters, then, are \ortho{\textel{*\wδ}} and \ortho{\textel{*δ\w}}.}
