\section{Homeric}\label{sec:Homeric}
\dpara{Intro} The Iliad of Homer never had a standard form, rather it varied between poets for the audience in attendance. Instead of learning the entire piece, these poets memorized prebuilt lines, which they used as a foundation to improvise the rest of the epic. These are the oldest lines of the Iliad, surviving from well before the first scribes recorded the poem. As such, they show irregularities stemming from historic scansion of static phrases.

\dpara{Apocope} Such irregularities often take the form of the vowels failing to elide across word boundaries. \bookref{Iliad}{15}{214} gives the example \greek{Ηφαίστοιο ἄνακτος},\autocite[XV.214]{Iliad_1999} without any apocope, or vowel elision, between the genitive ending \greek{-οιο} and the following alpha. In Greek, like in French and Italian, a word-terminal vowel will often elide before a word-initial vowel. That is to say, a vowel at the end of a word will drop if the following word begins with another vowel. Compare the elision in Greek  \greek{μεθ' ἡμῖν} ``with us'', from \greek{μετὰ ἡμῖν}, to French  \french{J’ai sommeil} ``I am tired'' $\gets$ \french{Je ai sommeil}. This implies that there was some phoneme, some distinct sound, was present at the start of the word \greek{ἄνακτος}. This phoneme was a consonant, which will be written as *\w\ until the exact sound can be determined, rather than a vowel, *\vowel, as a vowel would have still allowed elision. The sequence with a vowel would read \iform[G]{Ἡφαίσοιο \vowelάνακτος} to \iform[G]{Ἡφαίστοι' \vowelάνακτος} $\to$ \iform[G]{Ἡφαίστοι' ἄνακτος}. Note that an asterisk before any form means that it is unattested, meaning that it does not survive in the written record. As such, that form is hypothetical, based on the current understanding of Greek. A superscript \textit{x} before a form means that it is expected to be incorrect. Either it is a hypothetical form that does not fit with the current understanding of Greek, or it is a form that contradicts extant records.

\noindent\textcolor{red}{input [+consonant]}

\dpara{LBP}\dnote{2 ¶s?} Another type of historic scansion involves short vowels scanning as long. \bookref{Iliad}{5}{308}  has the phrase \greek{ἀπὸ ῥίνον}, with the omicron \ortho{\textel{ο}} scanning as long, even though the letter only represents a short vowel.\autocite[V.803]{Iliad_1999} This breaks rank with the traditional epic scansion, that is, the means of breaking a line of poetry into a series of short and long syllables. The Iliad, like most epics, uses a system called dactylic hexameter. Each line divides into six units, hence the \textit{hex-} from Greek \greek{ἕξ}\ ``six''. Each unit divides into three syllables -- one long followed by two shorts -- or two syllables -- two longs in a row. These resemble the shape of a finger, \textit{dactyl-} coming from Greek \greek{δάκτυλος} ``finger''. long syllables represent the first bone of the finger, while the short syllables represent the second and third bones. In any dactylic meter, a short vowel followed by one or no consonants creates a short syllable. A short vowel followed by two or more consonants creates a long syllable, regardless of if those consonants are in the same word, or across a word boundary. A long vowel creates a long syllable in any position. In this instance, the omicron \ortho{\textel{ο}} should scan as short, being a short vowel only followed by one consonant. Given that this omicron scans as long \ortho{\textel{ο}}, the following rho \ortho{\textel{ρ}} must be an element of a consonant cluster. The apocopic forms \greek{ἀπ'} and \greek{ἀφ'} disqualify the form \greek{ἀπό\w}, leaving only two possible positions for the hypothetical consonant: either before the rho, \ortho{\textel{*\wρ}}, or after it, \ortho{\textel{*ρ\w}}.

\dpara{LBP II} \dnote{Cut?} Affixes can also point to a similar process happening within a word boundary. Scansion of \bookref{Iliad}{1}{33} shows the word \greek{\Asm{ε}δεισεν},\autocite[I.33]{Iliad_1999} with a word-initial epsilon \greek{ἐ-} counting as a long vowel. In the aorist tense, or the ``simple past'' tense, the verb takes a prefix called the epsilon augment. As with the letter omicron, epsilon encodes a short vowel, implying that another consonant is missing. The possible clusters, then, are \ortho{\textel{*\wδ}} and \ortho{\textel{*δ\w}}.
