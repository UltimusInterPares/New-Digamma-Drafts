%%% Language Formatting

% Language Input and Display
\usepackage[greek, german, english]{babel}
   %\languageattribute{greek}{ancient}
   \babeltags{en = english}
   \babeltags{de = german}
   \babeltags{el = greek}
   
\usepackage {teubner}

%% Linguistic Characters

% Classical Characters:
% Digamma is pre-coded into teubner as \F\f or \Digamma\textdigamma
% Heta is pre-coded as \h
% Jod/jota is pre-coded as \j
% Semivowel marker for ι/υ is available \c or \semi
% Carons for palatalized consonants are encoded as \cap

% My Own Character Sets:
\newcommand{\w}{\texten{\.{c}}}
\newcommand{\W}{\texten{\.{C}}}

\newcommand{\vowel}{\texten{\.{v}}}
\newcommand{\Vowel}{\texten{\.{V}}}

\newcommand{\ash}{\texten{\={\ae}}}
\newcommand{\Ash}{\texten{\={\AE}}}

\usepackage{amssymb} % for nicer empty set
    \let\emptyset\varnothing
\usepackage{units} % for nicer fraction
    \let\frac\nicefrac
    
    \newcommand{\ea}{$\nicefrac{\textrm{\textel{ε}}}{\textrm{\textel{α}}}$}
    
% Greek Roots
\newcommand{\groot}[1]{$\sqrt{\textrm{\textel{#1}}}$}
\newcommand{\gstem}[1]{$\sqrt[stem]{\textrm{\textel{#1}}}$}

% Incorrect Forms:
	%Lets you choose Greek or Latin characters
		% Hopefully
\newcommand{\iform}[2][G]{%
  \ifx G#1 \bgroup\sffamily\texten{\textsuperscript{\tiny x}}\egroup\textel{#2}\else
  \ifx L#1 \bgroup\sffamily\texten{\textsuperscript{\tiny x}}\egroup\texten{#2}%
  \fi\fi
}
			% Well hot damn it works

% IPA:
\usepackage{tipa}
    \newcommand{\ipa}[1]{\textipa{#1}}
    
% Various Brackets
\usepackage{textcomp}
    \newcommand{\ortho}[1]{\texten{\textlangle}#1\texten{\textrangle}} % orthographic notation
    \newcommand{\phonet}[1]{[\ipa{#1}]} % phonetic notation
    \newcommand{\phonem}[1]{/\ipa{#1}/}
    
    \newcommand{\spell}[2]{\ortho{\textel{#1}}~\phonem{\ipa{#2}}}
    
%%% Historical Characters
\usepackage{greek6cbc}
\usepackage{greek4cbc}
\usepackage{etruscan}
\usepackage{runic}
\usepackage{linearb}
\usepackage{cypriot}
\usepackage{ugarite}
\usepackage{oldprsn}
